\documentclass[a4paper,12pt]{article}

\input{header.tex}

\title{Отчёт по лабораторной работе \\ <<Локальные сети>>}
\author{Фроловский Алексей Вадимлвич}

\begin{document}

\maketitle

\tableofcontents

% Текст отчёта должен быть читаемым!!! Написанное здесь является рыбой.

\section{Получение адреса по DHCP}

Получение "случайного" адреса можно продемонстрировать на примере
сети \textbf{lan2} с адресом \textbf{10.102.0.0/16}. Для этого
необходимо настроить службу DHCP на маршрутизаторе \textbf{r2}:
в файле \textbf{/etc/dhcp3/dhcpd.conf} следует указать следующие
настройки:
\begin{Verbatim}
subnet 172.16.0.0 netmask 255.255.0.0 \{\}

subnet 10.102.0.0 netmask 255.255.0.0
\{
  range 10.102.0.2 10.102.0.200;
  option routers 10.102.0.1;
  option domain-name-servers 192.168.0.1;
\}
\end{Verbatim}

Отметим, что в настройке \textbf{range} указывается диапазон,
в котором будут динамически выдаваться ip-адреса, в
\textbf{option routers} - ip-адреса маршрутизаторов сети,
\textbf{option domain-name-servers} - ip-адреса DNS-серверов.

Выполним на маршрутизаторе \textbf{r2}:
\begin{Verbatim}
tcpdump -tnv -i eth0 udp
\end{Verbatim}

Получим следующий вывод:
\begin{Verbatim}
IP (tos 0x10, ttl 128, id 0, offset 0, flags [none], proto UDP (17), length 328) 
	0.0.0.0.68 > 255.255.255.255.67: BOOTP/DHCP, Request from 10:10:10:10:10:ee, 
		length 300, xid 0x82701421, Flags [none]
          Client-Ethernet-Address 10:10:10:10:10:ee [|bootp]
IP (tos 0x10, ttl 128, id 0, offset 0, flags [none], proto UDP (17), length 328) 
	10.102.0.1.67 > 10.102.0.2.68: BOOTP/DHCP, Reply, length 300, xid 0x82701421, Flags [none]
          Your-IP 10.102.0.2
          Client-Ethernet-Address 10:10:10:10:10:ee [|bootp]
IP (tos 0x10, ttl 128, id 0, offset 0, flags [none], proto UDP (17), length 328) 
	0.0.0.0.68 > 255.255.255.255.67: BOOTP/DHCP, Request from 10:10:10:10:10:ee, 
		length 300, xid 0x82701421, Flags [none]
          Client-Ethernet-Address 10:10:10:10:10:ee [|bootp]
IP (tos 0x10, ttl 128, id 0, offset 0, flags [none], proto UDP (17), length 328) 
	10.102.0.1.67 > 10.102.0.2.68: BOOTP/DHCP, Reply, length 300, xid 0x82701421, Flags [none]
          Your-IP 10.102.0.2
          Client-Ethernet-Address 10:10:10:10:10:ee [|bootp]
IP (tos 0x0, ttl 64, id 0, offset 0, flags [DF], proto UDP (17), length 328) 
	10.102.0.2.68 > 10.102.0.1.67: BOOTP/DHCP, Request from 10:10:10:10:10:ee, 
		length 300, xid 0x7185b80d, Flags [none]
          Client-IP 10.102.0.2
          Client-Ethernet-Address 10:10:10:10:10:ee [|bootp]
IP (tos 0x10, ttl 128, id 0, offset 0, flags [none], proto UDP (17), length 328) 
	0.0.0.0.68 > 255.255.255.255.67: BOOTP/DHCP, Request from 10:10:10:10:10:ee, 
		length 300, xid 0xe433c82b, Flags [none]
          Client-Ethernet-Address 10:10:10:10:10:ee [|bootp]
IP (tos 0x10, ttl 128, id 0, offset 0, flags [none], proto UDP (17), length 328) 
	10.102.0.1.67 > 10.102.0.2.68: BOOTP/DHCP, Reply, length 300, xid 0xe433c82b, Flags [none]
          Your-IP 10.102.0.2
          Client-Ethernet-Address 10:10:10:10:10:ee [|bootp]
IP (tos 0x10, ttl 128, id 0, offset 0, flags [none], proto UDP (17), length 328) 
	0.0.0.0.68 > 255.255.255.255.67: BOOTP/DHCP, Request from 10:10:10:10:10:ee, 
		length 300, xid 0xe433c82b, Flags [none]
          Client-Ethernet-Address 10:10:10:10:10:ee [|bootp]
IP (tos 0x10, ttl 128, id 0, offset 0, flags [none], proto UDP (17), length 328) 
	10.102.0.1.67 > 10.102.0.2.68: BOOTP/DHCP, Reply, length 300, xid 0xe433c82b, Flags [none]
          Your-IP 10.102.0.2
          Client-Ethernet-Address 10:10:10:10:10:ee [|bootp]
\end{Verbatim}

Получение "фиксированных" адресов можно продемонстрировать на
примере сети \textbf{lan1} с адресом \textbf{10.0.0.0/16}. Для
этого необходимо настроить службу DHCP на маршрутизаторе
\textbf{r1}: в файле \textbf{/etc/dhcp3/dhcpd.conf} следует
указать следующие настройки:
\begin{Verbatim}
subnet 172.16.0.0 netmask 255.255.0.0 \{\}

subnet 10.0.0.0 netmask 255.255.0.0
\{
  range 10.0.0.2 10.0.10.200;
  option routers 10.0.0.1;
  option domain-name-servers 192.168.0.1;
\}

host ws11 \{
    hardware ethernet 10:10:10:10:10:BA;
    fixed-address 10.0.1.1;
\}

host ws12 \{
    hardware ethernet 10:10:10:10:10:BB;
    fixed-address 10.0.2.1;
\}

host ws13 \{
    hardware ethernet 10:10:10:10:10:BC;
    fixed-address 10.0.3.1;
\}

host s11 \{
    hardware ethernet 10:10:10:10:20:AA;
    fixed-address 10.0.4.10;
\}

host s12 \{
    hardware ethernet 10:10:10:10:20:BB;
    fixed-address 10.0.4.20;
\}
\end{Verbatim}

Выполним на маршрутизаторе \textbf{r2}:
\begin{Verbatim}
tcpdump -tnv -i eth0 udp
\end{Verbatim}

Получим следующий вывод:
\begin{Verbatim}
IP (tos 0x10, ttl 128, id 0, offset 0, flags [none], proto UDP (17), length 328) 
	0.0.0.0.68 > 255.255.255.255.67: BOOTP/DHCP, Request from 10:10:10:10:20:bb, 
		length 300, xid 0x28576d02, Flags [none]
	  Client-Ethernet-Address 10:10:10:10:20:bb [|bootp]
IP (tos 0x10, ttl 128, id 0, offset 0, flags [none], proto UDP (17), length 328) 
	10.0.0.1.67 > 10.0.4.20.68: BOOTP/DHCP, Reply, 
		length 300, xid 0x28576d02, Flags [none]
	  Your-IP 10.0.4.20
	  Client-Ethernet-Address 10:10:10:10:20:bb [|bootp]
IP (tos 0x10, ttl 128, id 0, offset 0, flags [none], proto UDP (17), length 328) 
	0.0.0.0.68 > 255.255.255.255.67: BOOTP/DHCP, Request from 10:10:10:10:20:bb, 
		length 300, xid 0x28576d02, Flags [none]
	  Client-Ethernet-Address 10:10:10:10:20:bb [|bootp]
IP (tos 0x10, ttl 128, id 0, offset 0, flags [none], proto UDP (17), length 328) 
	10.0.0.1.67 > 10.0.4.20.68: BOOTP/DHCP, Reply, length 300, xid 0x28576d02, Flags [none]
	  Your-IP 10.0.4.20
	  Client-Ethernet-Address 10:10:10:10:20:bb [|bootp]
IP (tos 0x0, ttl 64, id 0, offset 0, flags [DF], proto UDP (17), length 328) 
	10.0.4.10.68 > 10.0.0.1.67: BOOTP/DHCP, Request from 10:10:10:10:20:aa, 
		length 300, xid 0xc48d122e, Flags [none]
	  Client-IP 10.0.4.10
	  Client-Ethernet-Address 10:10:10:10:20:aa [|bootp]
IP (tos 0x10, ttl 128, id 0, offset 0, flags [none], proto UDP (17), length 328) 
	0.0.0.0.68 > 255.255.255.255.67: BOOTP/DHCP, Request from 10:10:10:10:10:ba, 
		length 300, xid 0xad5a342d, Flags [none]
	  Client-Ethernet-Address 10:10:10:10:10:ba [|bootp]
IP (tos 0x10, ttl 128, id 0, offset 0, flags [none], proto UDP (17), length 328) 
	10.0.0.1.67 > 10.0.1.1.68: BOOTP/DHCP, Reply, length 300, xid 0xad5a342d, Flags [none]
	  Your-IP 10.0.1.1
	  Client-Ethernet-Address 10:10:10:10:10:ba [|bootp]
IP (tos 0x10, ttl 128, id 0, offset 0, flags [none], proto UDP (17), length 328) 
	0.0.0.0.68 > 255.255.255.255.67: BOOTP/DHCP, Request from 10:10:10:10:10:ba, 
		length 300, xid 0xad5a342d, Flags [none]
	  Client-Ethernet-Address 10:10:10:10:10:ba [|bootp]
IP (tos 0x10, ttl 128, id 0, offset 0, flags [none], proto UDP (17), length 328) 
	10.0.0.1.67 > 10.0.1.1.68: BOOTP/DHCP, Reply, length 300, xid 0xad5a342d, Flags [none]
	  Your-IP 10.0.1.1
	  Client-Ethernet-Address 10:10:10:10:10:ba [|bootp]
IP (tos 0x10, ttl 128, id 0, offset 0, flags [none], proto UDP (17), length 328) 
	0.0.0.0.68 > 255.255.255.255.67: BOOTP/DHCP, Request from 10:10:10:10:10:bb, 
		length 300, xid 0xa001b35a, Flags [none]
	  Client-Ethernet-Address 10:10:10:10:10:bb [|bootp]
IP (tos 0x10, ttl 128, id 0, offset 0, flags [none], proto UDP (17), length 328) 
	10.0.0.1.67 > 10.0.2.1.68: BOOTP/DHCP, Reply, length 300, xid 0xa001b35a, Flags [none]
	  Your-IP 10.0.2.1
	  Client-Ethernet-Address 10:10:10:10:10:bb [|bootp]
IP (tos 0x10, ttl 128, id 0, offset 0, flags [none], proto UDP (17), length 328) 
	0.0.0.0.68 > 255.255.255.255.67: BOOTP/DHCP, Request from 10:10:10:10:10:bb, 
		length 300, xid 0xa001b35a, Flags [none]
	  Client-Ethernet-Address 10:10:10:10:10:bb [|bootp]
IP (tos 0x10, ttl 128, id 0, offset 0, flags [none], proto UDP (17), length 328) 
	10.0.0.1.67 > 10.0.2.1.68: BOOTP/DHCP, Reply, length 300, xid 0xa001b35a, Flags [none]
	  Your-IP 10.0.2.1
	  Client-Ethernet-Address 10:10:10:10:10:bb [|bootp]
IP (tos 0x10, ttl 128, id 0, offset 0, flags [none], proto UDP (17), length 328) 
	0.0.0.0.68 > 255.255.255.255.67: BOOTP/DHCP, Request from 10:10:10:10:20:bb, 
		length 300, xid 0x93a8a447, Flags [none]
	  Client-Ethernet-Address 10:10:10:10:20:bb [|bootp]
IP (tos 0x10, ttl 128, id 0, offset 0, flags [none], proto UDP (17), length 328) 
	10.0.0.1.67 > 10.0.4.20.68: BOOTP/DHCP, Reply, length 300, xid 0x93a8a447, Flags [none]
	  Your-IP 10.0.4.20
	  Client-Ethernet-Address 10:10:10:10:20:bb [|bootp]
IP (tos 0x10, ttl 128, id 0, offset 0, flags [none], proto UDP (17), length 328) 
	0.0.0.0.68 > 255.255.255.255.67: BOOTP/DHCP, Request from 10:10:10:10:20:bb, 
		length 300, xid 0x93a8a447, Flags [none]
	  Client-Ethernet-Address 10:10:10:10:20:bb [|bootp]
IP (tos 0x10, ttl 128, id 0, offset 0, flags [none], proto UDP (17), length 328) 
	10.0.0.1.67 > 10.0.4.20.68: BOOTP/DHCP, Reply, length 300, xid 0x93a8a447, Flags [none]
	  Your-IP 10.0.4.20
	  Client-Ethernet-Address 10:10:10:10:20:bb [|bootp]
IP (tos 0x0, ttl 64, id 0, offset 0, flags [DF], proto UDP (17), length 328) 
	10.0.1.1.68 > 10.0.0.1.67: BOOTP/DHCP, Request from 10:10:10:10:10:ba, 
		length 300, xid 0x437a631f, Flags [none]
	  Client-IP 10.0.1.1
	  Client-Ethernet-Address 10:10:10:10:10:ba [|bootp]
IP (tos 0x10, ttl 128, id 0, offset 0, flags [none], proto UDP (17), length 328) 
	0.0.0.0.68 > 255.255.255.255.67: BOOTP/DHCP, Request from 10:10:10:10:10:bc, 
		length 300, xid 0xecf5c126, Flags [none]
	  Client-Ethernet-Address 10:10:10:10:10:bc [|bootp]
IP (tos 0x10, ttl 128, id 0, offset 0, flags [none], proto UDP (17), length 328) 
	10.0.0.1.67 > 10.0.3.1.68: BOOTP/DHCP, Reply, length 300, xid 0xecf5c126, Flags [none]
	  Your-IP 10.0.3.1
	  Client-Ethernet-Address 10:10:10:10:10:bc [|bootp]
IP (tos 0x10, ttl 128, id 0, offset 0, flags [none], proto UDP (17), length 328) 
	0.0.0.0.68 > 255.255.255.255.67: BOOTP/DHCP, Request from 10:10:10:10:10:bc, 
		length 300, xid 0xecf5c126, Flags [none]
	  Client-Ethernet-Address 10:10:10:10:10:bc [|bootp]
IP (tos 0x10, ttl 128, id 0, offset 0, flags [none], proto UDP (17), length 328) 
	10.0.0.1.67 > 10.0.3.1.68: BOOTP/DHCP, Reply, length 300, xid 0xecf5c126, Flags [none]
	  Your-IP 10.0.3.1
	  Client-Ethernet-Address 10:10:10:10:10:bc [|bootp]
IP (tos 0x10, ttl 128, id 0, offset 0, flags [none], proto UDP (17), length 328) 
	0.0.0.0.68 > 255.255.255.255.67: BOOTP/DHCP, Request from 10:10:10:10:10:ba, 
		length 300, xid 0x58752e33, Flags [none]
	  Client-Ethernet-Address 10:10:10:10:10:ba [|bootp]
IP (tos 0x10, ttl 128, id 0, offset 0, flags [none], proto UDP (17), length 328) 
	10.0.0.1.67 > 10.0.1.1.68: BOOTP/DHCP, Reply, length 300, xid 0x58752e33, Flags [none]
	  Your-IP 10.0.1.1
	  Client-Ethernet-Address 10:10:10:10:10:ba [|bootp]
IP (tos 0x10, ttl 128, id 0, offset 0, flags [none], proto UDP (17), length 328) 
	0.0.0.0.68 > 255.255.255.255.67: BOOTP/DHCP, Request from 10:10:10:10:10:ba, 
		length 300, xid 0x58752e33, Flags [none]
	  Client-Ethernet-Address 10:10:10:10:10:ba [|bootp]
IP (tos 0x10, ttl 128, id 0, offset 0, flags [none], proto UDP (17), length 328) 
	10.0.0.1.67 > 10.0.1.1.68: BOOTP/DHCP, Reply, length 300, xid 0x58752e33, Flags [none]
	  Your-IP 10.0.1.1
	  Client-Ethernet-Address 10:10:10:10:10:ba [|bootp]
IP (tos 0x0, ttl 64, id 0, offset 0, flags [DF], proto UDP (17), length 328) 
	10.0.2.1.68 > 10.0.0.1.67: BOOTP/DHCP, Request from 10:10:10:10:10:bb, 
		length 300, xid 0x193e306b, Flags [none]
	  Client-IP 10.0.2.1
	  Client-Ethernet-Address 10:10:10:10:10:bb [|bootp]
IP (tos 0x10, ttl 128, id 0, offset 0, flags [none], proto UDP (17), length 328) 
	0.0.0.0.68 > 255.255.255.255.67: BOOTP/DHCP, Request from 10:10:10:10:10:bb, 
		length 300, xid 0x8511b67e, Flags [none]
	  Client-Ethernet-Address 10:10:10:10:10:bb [|bootp]
IP (tos 0x10, ttl 128, id 0, offset 0, flags [none], proto UDP (17), length 328) 
	10.0.0.1.67 > 10.0.2.1.68: BOOTP/DHCP, Reply, length 300, xid 0x8511b67e, Flags [none]
	  Your-IP 10.0.2.1
	  Client-Ethernet-Address 10:10:10:10:10:bb [|bootp]
IP (tos 0x10, ttl 128, id 0, offset 0, flags [none], proto UDP (17), length 328) 
	0.0.0.0.68 > 255.255.255.255.67: BOOTP/DHCP, Request from 10:10:10:10:10:bb, 
		length 300, xid 0x8511b67e, Flags [none]
	  Client-Ethernet-Address 10:10:10:10:10:bb [|bootp]
IP (tos 0x10, ttl 128, id 0, offset 0, flags [none], proto UDP (17), length 328) 
	10.0.0.1.67 > 10.0.2.1.68: BOOTP/DHCP, Reply, length 300, xid 0x8511b67e, Flags [none]
	  Your-IP 10.0.2.1
	  Client-Ethernet-Address 10:10:10:10:10:bb [|bootp]
IP (tos 0x10, ttl 128, id 0, offset 0, flags [none], proto UDP (17), length 328) 
	0.0.0.0.68 > 255.255.255.255.67: BOOTP/DHCP, Request from 10:10:10:10:20:aa, 
		length 300, xid 0xd215445, Flags [none]
	  Client-Ethernet-Address 10:10:10:10:20:aa [|bootp]
IP (tos 0x10, ttl 128, id 0, offset 0, flags [none], proto UDP (17), length 328) 
	10.0.0.1.67 > 10.0.4.10.68: BOOTP/DHCP, Reply, length 300, xid 0xd215445, Flags [none]
	  Your-IP 10.0.4.10
	  Client-Ethernet-Address 10:10:10:10:20:aa [|bootp]
IP (tos 0x10, ttl 128, id 0, offset 0, flags [none], proto UDP (17), length 328) 
	0.0.0.0.68 > 255.255.255.255.67: BOOTP/DHCP, Request from 10:10:10:10:20:aa, 
		length 300, xid 0xd215445, Flags [none]
	  Client-Ethernet-Address 10:10:10:10:20:aa [|bootp]
IP (tos 0x10, ttl 128, id 0, offset 0, flags [none], proto UDP (17), length 328) 
	10.0.0.1.67 > 10.0.4.10.68: BOOTP/DHCP, Reply, length 300, xid 0xd215445, Flags [none]
	  Your-IP 10.0.4.10
	  Client-Ethernet-Address 10:10:10:10:20:aa [|bootp]
IP (tos 0x10, ttl 128, id 0, offset 0, flags [none], proto UDP (17), length 328) 
	0.0.0.0.68 > 255.255.255.255.67: BOOTP/DHCP, Request from 10:10:10:10:10:bc, 
		length 300, xid 0x3dad9929, Flags [none]
	  Client-Ethernet-Address 10:10:10:10:10:bc [|bootp]
IP (tos 0x10, ttl 128, id 0, offset 0, flags [none], proto UDP (17), length 328) 
	10.0.0.1.67 > 10.0.3.1.68: BOOTP/DHCP, Reply, length 300, xid 0x3dad9929, Flags [none]
	  Your-IP 10.0.3.1
	  Client-Ethernet-Address 10:10:10:10:10:bc [|bootp]
IP (tos 0x10, ttl 128, id 0, offset 0, flags [none], proto UDP (17), length 328) 
	0.0.0.0.68 > 255.255.255.255.67: BOOTP/DHCP, Request from 10:10:10:10:10:bc, 
		length 300, xid 0x3dad9929, Flags [none]
	  Client-Ethernet-Address 10:10:10:10:10:bc [|bootp]
IP (tos 0x10, ttl 128, id 0, offset 0, flags [none], proto UDP (17), length 328) 
	10.0.0.1.67 > 10.0.3.1.68: BOOTP/DHCP, Reply, length 300, xid 0x3dad9929, Flags [none]
	  Your-IP 10.0.3.1
	  Client-Ethernet-Address 10:10:10:10:10:bc [|bootp]
IP (tos 0x10, ttl 128, id 0, offset 0, flags [none], proto UDP (17), length 328) 
	0.0.0.0.68 > 255.255.255.255.67: BOOTP/DHCP, Request from 10:10:10:10:20:aa, 
		length 300, xid 0xd3cd0b29, Flags [none]
	  Client-Ethernet-Address 10:10:10:10:20:aa [|bootp]
IP (tos 0x10, ttl 128, id 0, offset 0, flags [none], proto UDP (17), length 328) 
	10.0.0.1.67 > 10.0.4.10.68: BOOTP/DHCP, Reply, length 300, xid 0xd3cd0b29, Flags [none]
	  Your-IP 10.0.4.10
	  Client-Ethernet-Address 10:10:10:10:20:aa [|bootp]
IP (tos 0x10, ttl 128, id 0, offset 0, flags [none], proto UDP (17), length 328) 
	0.0.0.0.68 > 255.255.255.255.67: BOOTP/DHCP, Request from 10:10:10:10:20:aa, 
		length 300, xid 0xd3cd0b29, Flags [none]
	  Client-Ethernet-Address 10:10:10:10:20:aa [|bootp]
IP (tos 0x10, ttl 128, id 0, offset 0, flags [none], proto UDP (17), length 328) 
	10.0.0.1.67 > 10.0.4.10.68: BOOTP/DHCP, Reply, length 300, xid 0xd3cd0b29, Flags [none]
	  Your-IP 10.0.4.10
	  Client-Ethernet-Address 10:10:10:10:20:aa [|bootp]
\end{Verbatim}


\section{Использование VPN}

Для создания виртуальной частной сети используется служба OpenVPN.
В качестве сервера выступает маршрутизатор \textbf{r1}, ожидающего подключение, а в качестве клиента - \textbf{r2}. Сервер управляет
настройками VPN, поэтому на маршрутизаторе \textbf{r1} следует
изменить файл \textbf{/etc/openvpn/tun0.conf} следующим образом:
\begin{Verbatim}
local 172.16.1.3
proto udp
port 1194
dev tun

ifconfig 10.200.1.1 10.200.1.2
secret /etc/openvpn/keys/somesecret.key
status /var/log/openvpn/tun0.status
log /var/log/openvpn/tun0.log
\end{Verbatim}

А на маршрутизаторе \textbf{r2} этот файл следует отредактировать
следующим образом:
\begin{Verbatim}
remote 172.16.1.3 1194 
proto udp
dev tun

ifconfig 10.200.1.2 10.200.1.1
secret /etc/openvpn/keys/somesecret.key
status /var/log/openvpn/tun0.status
log /var/log/openvpn/tun0.log
\end{Verbatim}

Выполним \textbf{ip r} на маршрутизаторе \textbf{r1}:
\begin{Verbatim}
10.200.1.2 dev tun0  proto kernel  scope link  src 10.200.1.1 
10.0.0.0/16 dev eth0  proto kernel  scope link  src 10.0.0.1 
10.102.0.0/16 via 10.200.1.2 dev tun0  proto zebra  metric 2 
172.16.0.0/16 dev eth1  proto kernel  scope link  src 172.16.1.3 
default via 172.16.1.2 dev eth1 
\end{Verbatim}

Выполним \textbf{ip r} на маршрутизаторе \textbf{r2}:
\begin{Verbatim}
10.200.1.1 dev tun0  proto kernel  scope link  src 10.200.1.2 
10.0.0.0/16 via 10.200.1.1 dev tun0  proto zebra  metric 2 
10.102.0.0/16 dev eth0  proto kernel  scope link  src 10.102.0.1 
172.16.0.0/16 dev eth1  proto kernel  scope link  src 172.16.1.4 
default via 172.16.1.2 dev eth1
\end{Verbatim}

Выполним \textbf{ip -4 a} на маршрутизаторе \textbf{r1}:
\begin{Verbatim}
1: lo: <LOOPBACK,UP,LOWER_UP> mtu 16436 qdisc noqueue 
    inet 127.0.0.1/8 scope host lo
3: eth1: <BROADCAST,MULTICAST,UP,LOWER_UP> mtu 1500 qdisc pfifo_fast qlen 1000
    inet 172.16.1.3/16 brd 172.16.255.255 scope global eth1
4: eth0: <BROADCAST,MULTICAST,UP,LOWER_UP> mtu 1500 qdisc pfifo_fast qlen 1000
    inet 10.0.0.1/16 brd 10.0.255.255 scope global eth0
5: tun0: <POINTOPOINT,MULTICAST,NOARP,UP,LOWER_UP> mtu 1500 qdisc pfifo_fast qlen 100
    inet 10.200.1.1 peer 10.200.1.2/32 scope global tun0
\end{Verbatim}

Выполним \textbf{ip -4 a} на маршрутизаторе \textbf{r2}:
\begin{Verbatim}
1: lo: <LOOPBACK,UP,LOWER_UP> mtu 16436 qdisc noqueue 
    inet 127.0.0.1/8 scope host lo
3: eth1: <BROADCAST,MULTICAST,UP,LOWER_UP> mtu 1500 qdisc pfifo_fast qlen 1000
    inet 172.16.1.4/16 brd 172.16.255.255 scope global eth1
4: eth0: <BROADCAST,MULTICAST,UP,LOWER_UP> mtu 1500 qdisc pfifo_fast qlen 1000
    inet 10.102.0.1/16 brd 10.102.255.255 scope global eth0
5: tun0: <POINTOPOINT,MULTICAST,NOARP,UP,LOWER_UP> mtu 1500 qdisc pfifo_fast qlen 100
    inet 10.200.1.2 peer 10.200.1.1/32 scope global tun0
\end{Verbatim}

Выполним \textbf{tcpdump -tvn -i tun0 -s 1518 udp} на маршрутизаторе
\textbf{r1}:
\begin{Verbatim}
IP (tos 0x0, ttl 1, id 0, offset 0, flags [DF], proto UDP (17), length 52) 10.200.1.2.520 > 224.0.0.9.520: 
	RIPv2, Response, length: 24, routes: 1
	  AFI: IPv4:      10.102.0.0/16, tag 0x0000, metric: 1, next-hop: self
IP (tos 0x0, ttl 1, id 0, offset 0, flags [DF], proto UDP (17), length 52) 10.200.1.1.520 > 224.0.0.9.520: 
	RIPv2, Response, length: 24, routes: 1
	  AFI: IPv4:        10.0.0.0/16, tag 0x0000, metric: 1, next-hop: self
IP (tos 0x0, ttl 1, id 0, offset 0, flags [DF], proto UDP (17), length 52) 10.200.1.2.520 > 224.0.0.9.520: 
	RIPv2, Response, length: 24, routes: 1
	  AFI: IPv4:      10.102.0.0/16, tag 0x0000, metric: 1, next-hop: self
IP (tos 0x0, ttl 1, id 0, offset 0, flags [DF], proto UDP (17), length 52) 10.200.1.1.520 > 224.0.0.9.520: 
	RIPv2, Response, length: 24, routes: 1
	  AFI: IPv4:        10.0.0.0/16, tag 0x0000, metric: 1, next-hop: self
IP (tos 0x0, ttl 1, id 0, offset 0, flags [DF], proto UDP (17), length 52) 10.200.1.2.520 > 224.0.0.9.520: 
	RIPv2, Response, length: 24, routes: 1
	  AFI: IPv4:      10.102.0.0/16, tag 0x0000, metric: 1, next-hop: self
IP (tos 0x0, ttl 1, id 0, offset 0, flags [DF], proto UDP (17), length 52) 10.200.1.1.520 > 224.0.0.9.520: 
	RIPv2, Response, length: 24, routes: 1
	  AFI: IPv4:        10.0.0.0/16, tag 0x0000, metric: 1, next-hop: self
IP (tos 0x0, ttl 1, id 0, offset 0, flags [DF], proto UDP (17), length 52) 10.200.1.2.520 > 224.0.0.9.520: 
	RIPv2, Response, length: 24, routes: 1
	  AFI: IPv4:      10.102.0.0/16, tag 0x0000, metric: 1, next-hop: self
IP (tos 0x0, ttl 1, id 0, offset 0, flags [DF], proto UDP (17), length 52) 10.200.1.1.520 > 224.0.0.9.520: 
	RIPv2, Response, length: 24, routes: 1
	  AFI: IPv4:        10.0.0.0/16, tag 0x0000, metric: 1, next-hop: self
IP (tos 0x0, ttl 1, id 0, offset 0, flags [DF], proto UDP (17), length 52) 10.200.1.2.520 > 224.0.0.9.520: 
	RIPv2, Response, length: 24, routes: 1
	  AFI: IPv4:      10.102.0.0/16, tag 0x0000, metric: 1, next-hop: self
IP (tos 0x0, ttl 1, id 0, offset 0, flags [DF], proto UDP (17), length 52) 10.200.1.1.520 > 224.0.0.9.520: 
	RIPv2, Response, length: 24, routes: 1
	  AFI: IPv4:        10.0.0.0/16, tag 0x0000, metric: 1, next-hop: self
IP (tos 0x0, ttl 1, id 0, offset 0, flags [DF], proto UDP (17), length 52) 10.200.1.2.520 > 224.0.0.9.520: 
	RIPv2, Response, length: 24, routes: 1
	  AFI: IPv4:      10.102.0.0/16, tag 0x0000, metric: 1, next-hop: self
IP (tos 0x0, ttl 1, id 0, offset 0, flags [DF], proto UDP (17), length 52) 10.200.1.1.520 > 224.0.0.9.520: 
	RIPv2, Response, length: 24, routes: 1
	  AFI: IPv4:        10.0.0.0/16, tag 0x0000, metric: 1, next-hop: self
IP (tos 0x0, ttl 1, id 0, offset 0, flags [DF], proto UDP (17), length 52) 10.200.1.2.520 > 224.0.0.9.520: 
	RIPv2, Response, length: 24, routes: 1
	  AFI: IPv4:      10.102.0.0/16, tag 0x0000, metric: 1, next-hop: self
IP (tos 0x0, ttl 1, id 0, offset 0, flags [DF], proto UDP (17), length 52) 10.200.1.1.520 > 224.0.0.9.520: 
	RIPv2, Response, length: 24, routes: 1
	  AFI: IPv4:        10.0.0.0/16, tag 0x0000, metric: 1, next-hop: self
IP (tos 0x0, ttl 1, id 0, offset 0, flags [DF], proto UDP (17), length 52) 10.200.1.2.520 > 224.0.0.9.520: 
	RIPv2, Response, length: 24, routes: 1
	  AFI: IPv4:      10.102.0.0/16, tag 0x0000, metric: 1, next-hop: self
IP (tos 0x0, ttl 1, id 0, offset 0, flags [DF], proto UDP (17), length 52) 10.200.1.1.520 > 224.0.0.9.520: 
	RIPv2, Response, length: 24, routes: 1
	  AFI: IPv4:        10.0.0.0/16, tag 0x0000, metric: 1, next-hop: self
IP (tos 0x0, ttl 1, id 0, offset 0, flags [DF], proto UDP (17), length 52) 10.200.1.2.520 > 224.0.0.9.520: 
	RIPv2, Response, length: 24, routes: 1
	  AFI: IPv4:      10.102.0.0/16, tag 0x0000, metric: 1, next-hop: self
IP (tos 0x0, ttl 1, id 0, offset 0, flags [DF], proto UDP (17), length 52) 10.200.1.1.520 > 224.0.0.9.520: 
	RIPv2, Response, length: 24, routes: 1
	  AFI: IPv4:        10.0.0.0/16, tag 0x0000, metric: 1, next-hop: self
IP (tos 0x0, ttl 1, id 0, offset 0, flags [DF], proto UDP (17), length 52) 10.200.1.2.520 > 224.0.0.9.520: 
	RIPv2, Response, length: 24, routes: 1
	  AFI: IPv4:      10.102.0.0/16, tag 0x0000, metric: 1, next-hop: self
IP (tos 0x0, ttl 1, id 0, offset 0, flags [DF], proto UDP (17), length 52) 10.200.1.1.520 > 224.0.0.9.520: 
	RIPv2, Response, length: 24, routes: 1
	  AFI: IPv4:        10.0.0.0/16, tag 0x0000, metric: 1, next-hop: self
IP (tos 0x0, ttl 1, id 0, offset 0, flags [DF], proto UDP (17), length 52) 10.200.1.2.520 > 224.0.0.9.520: 
	RIPv2, Response, length: 24, routes: 1
	  AFI: IPv4:      10.102.0.0/16, tag 0x0000, metric: 1, next-hop: self
IP (tos 0x0, ttl 1, id 0, offset 0, flags [DF], proto UDP (17), length 52) 10.200.1.1.520 > 224.0.0.9.520: 
	RIPv2, Response, length: 24, routes: 1
	  AFI: IPv4:        10.0.0.0/16, tag 0x0000, metric: 1, next-hop: self
\end{Verbatim}


\section{Правила фильтации пакетов и трансляции пдресов}

Сценарий фильтрации на маршрутизаторе \textbf{r1}:

\begin{Verbatim}
#!/bin/sh
LAN=eth0
INET=eth1
VPN=tun0

# Удаление всех правил в таблице "filter" (по-умолчанию).
iptables -F

# Удаление правил в таблице "nat" (её надо указать явно).
iptables -F -t nat

# По-умолчанию все маршрутизируемые пакеты выбрасываются.
iptables --policy FORWARD DROP

# Для s11 разрешаем входящие соединения по smtp (порт 25)
iptables -t nat -A PREROUTING -p tcp --dport 25 -i $INET -j DNAT --to 10.0.4.10:25
iptables -A FORWARD -i $LAN -s 10.0.4.10 -p tcp -j ACCEPT
iptables -A FORWARD -i $INET -d 10.0.4.10 -p tcp --dport 25 -j ACCEPT

# Для s12 разрешаем входящие соелинения по http (порт 80)
iptables -t nat -A PREROUTING -p tcp --dport 80 -i $INET -j DNAT --to 10.0.4.20:80
iptables -A FORWARD -i $LAN -s 10.0.4.20 -p tcp -j ACCEPT
iptables -A FORWARD -i $INET -d 10.0.4.20 -p tcp --dport 80 -j ACCEPT

# Разрешаем любую маршрутизацию для интерфейса VPN
iptables -A FORWARD -i $VPN -j ACCEPT
iptables -A FORWARD -o $VPN -j ACCEPT

# Для ws11
# разрешаем сеть МГТУ
iptables -A FORWARD -i $LAN -s 10.0.1.1 -d 172.168.0.0/16 -p tcp --dport 80 -j ACCEPT
# запрещаем vkontakte.ru
iptables -A FORWARD -i $LAN -s 10.0.1.1 -d 87.240.156.160/27 -p tcp --dport 80 -j DROP
# запрещаем vk.com
iptables -A FORWARD -i $LAN -s 10.0.1.1 -d 87.240.131.96/27 -p tcp --dport 80 -j DROP
# разрешаем все остальные http запросы
iptables -A FORWARD -i $LAN -s 10.0.1.1 -p tcp --dport 80 -j ACCEPT

# Для ws12
# разрешаем rambler.ru
iptables -A FORWARD -i $LAN -s 10.0.2.1 -d 81.19.70.3 -p tcp --dport 80 -j ACCEPT
iptables -A FORWARD -i $LAN -s 10.0.2.1 -d 81.19.70.3 -p tcp --dport 443 -j ACCEPT
# разрешаем mail.ru
iptables -A FORWARD -i $LAN -s 10.0.2.1 -d 217.69.139.192/28 -p tcp --dport 80 -j ACCEPT
iptables -A FORWARD -i $LAN -s 10.0.2.1 -d 94.100.180.192/28 -p tcp --dport 80 -j ACCEPT
iptables -A FORWARD -i $LAN -s 10.0.2.1 -d 217.69.139.192/28 -p tcp --dport 443 -j ACCEPT
iptables -A FORWARD -i $LAN -s 10.0.2.1 -d 94.100.180.192/28 -p tcp --dport 443 -j ACCEPT

# Для ws13
# разрешаем сеть МГТУ
iptables -A FORWARD -i $LAN -s 10.0.3.1 -d 172.168.0.0/16 -p tcp -j ACCEPT
# разрешаем порты 80, 110, 443
iptables -A FORWARD -i $LAN -s 10.0.3.1 -p tcp --dport 80 -j ACCEPT
iptables -A FORWARD -i $LAN -s 10.0.3.1 -p tcp --dport 110 -j ACCEPT
iptables -A FORWARD -i $LAN -s 10.0.3.1 -p tcp --dport 443 -j ACCEPT

# Включение SNAT для маршрутизируемых пакетов, выходящих
# через eth1. Это правило выполняется после самой маршрутизации
# (POSTROUTING) и помещается в таблицу правил "nat".
iptables -t nat -A POSTROUTING -o $INET -j MASQUERADE
# Разрешение пакетов-ответов (они отслеживаются как 
# -- state ESTABLISHED)
iptables -A FORWARD -m state --state ESTABLISHED -i $INET -j ACCEPT
\end{Verbatim}

Выполним \textbf{iptables -L -nv} на \textbf{r1}:
\begin{Verbatim}
Chain INPUT (policy ACCEPT 2892 packets, 236K bytes)
 pkts bytes target     prot opt in     out     source               destination         

Chain FORWARD (policy DROP 0 packets, 0 bytes)
 pkts bytes target     prot opt in     out     source               destination         
    0     0 ACCEPT     tcp  --  eth0   *       10.0.4.10            0.0.0.0/0           
    0     0 ACCEPT     tcp  --  eth1   *       0.0.0.0/0            10.0.4.10           tcp dpt:25 
    0     0 ACCEPT     tcp  --  eth0   *       10.0.4.20            0.0.0.0/0           
    0     0 ACCEPT     tcp  --  eth1   *       0.0.0.0/0            10.0.4.20           tcp dpt:80 
    0     0 ACCEPT     all  --  tun0   *       0.0.0.0/0            0.0.0.0/0           
    0     0 ACCEPT     all  --  *      tun0    0.0.0.0/0            0.0.0.0/0           
    0     0 ACCEPT     tcp  --  eth0   *       10.0.1.1             172.168.0.0/16      tcp dpt:80 
    0     0 DROP       tcp  --  eth0   *       10.0.1.1             87.240.156.160/27   tcp dpt:80 
    0     0 DROP       tcp  --  eth0   *       10.0.1.1             87.240.131.96/27    tcp dpt:80 
    0     0 ACCEPT     tcp  --  eth0   *       10.0.1.1             0.0.0.0/0           tcp dpt:80 
    0     0 ACCEPT     tcp  --  eth0   *       10.0.2.1             81.19.70.3          tcp dpt:80 
    0     0 ACCEPT     tcp  --  eth0   *       10.0.2.1             81.19.70.3          tcp dpt:443 
    0     0 ACCEPT     tcp  --  eth0   *       10.0.2.1             217.69.139.192/28   tcp dpt:80 
    0     0 ACCEPT     tcp  --  eth0   *       10.0.2.1             94.100.180.192/28   tcp dpt:80 
    0     0 ACCEPT     tcp  --  eth0   *       10.0.2.1             217.69.139.192/28   tcp dpt:443 
    0     0 ACCEPT     tcp  --  eth0   *       10.0.2.1             94.100.180.192/28   tcp dpt:443 
    0     0 ACCEPT     tcp  --  eth0   *       10.0.3.1             172.168.0.0/16      
    0     0 ACCEPT     tcp  --  eth0   *       10.0.3.1             0.0.0.0/0           tcp dpt:80 
    0     0 ACCEPT     tcp  --  eth0   *       10.0.3.1             0.0.0.0/0           tcp dpt:110 
    0     0 ACCEPT     tcp  --  eth0   *       10.0.3.1             0.0.0.0/0           tcp dpt:443 
    0     0 ACCEPT     all  --  eth1   *       0.0.0.0/0            0.0.0.0/0           state ESTABLISHED 

Chain OUTPUT (policy ACCEPT 96 packets, 7380 bytes)
 pkts bytes target     prot opt in     out     source               destination
\end{Verbatim}

Выполним \textbf{iptables -L -nv -t nat} на \textbf{r1}:
\begin{Verbatim}
Chain PREROUTING (policy ACCEPT 254 packets, 35657 bytes)
 pkts bytes target     prot opt in     out     source               destination         
    0     0 DNAT       tcp  --  eth1   *       0.0.0.0/0            0.0.0.0/0           tcp dpt:25 to:10.0.4.10:25 
    0     0 DNAT       tcp  --  eth1   *       0.0.0.0/0            0.0.0.0/0           tcp dpt:80 to:10.0.4.20:80 

Chain POSTROUTING (policy ACCEPT 15 packets, 836 bytes)
 pkts bytes target     prot opt in     out     source               destination         
    0     0 MASQUERADE  all  --  *      eth1    0.0.0.0/0            0.0.0.0/0           

Chain OUTPUT (policy ACCEPT 15 packets, 836 bytes)
 pkts bytes target     prot opt in     out     source               destination
\end{Verbatim}

Сценарий фильтрации на маршрутизаторе \textbf{r2}:

\begin{Verbatim}
#!/bin/sh
LAN=eth0
INET=eth1
VPN=tun0

# Удаление всех правил в таблице "filter" (по-умолчанию).
iptables -F

# Удаление правил в таблице "nat" (её надо указать явно).
iptables -F -t nat

# По-умолчанию все маршрутизируемые пакеты выбрасываются.
iptables --policy FORWARD DROP

# Разрешаем любую маршрутизацию для интерфейса VPN
iptables -A FORWARD -i $VPN -j ACCEPT
iptables -A FORWARD -o $VPN -j ACCEPT

# Для pc21
# разрешаем порт 80
iptables -A FORWARD -o $INET -p tcp --dport 80 -j ACCEPT

# Включение SNAT для маршрутизируемых пакетов, выходящих
# через eth1. Это правило выполняется после самой маршрутизации
# (POSTROUTING) и помещается в таблицу правил "nat".
iptables -t nat -A POSTROUTING -o $INET -j MASQUERADE
# Разрешение пакетов-ответов (они отслеживаются как
# -- state ESTABLISHED)
iptables -A FORWARD -m state --state ESTABLISHED -i $INET -j ACCEPT
\end{Verbatim}

Выполним \textbf{iptables -L -nv} на \textbf{r2}:
\begin{Verbatim}
Chain INPUT (policy ACCEPT 3027 packets, 242K bytes)
 pkts bytes target     prot opt in     out     source               destination         

Chain FORWARD (policy DROP 0 packets, 0 bytes)
 pkts bytes target     prot opt in     out     source               destination         
    0     0 ACCEPT     all  --  tun0   *       0.0.0.0/0            0.0.0.0/0           
    0     0 ACCEPT     all  --  *      tun0    0.0.0.0/0            0.0.0.0/0           
    0     0 ACCEPT     tcp  --  *      eth1    0.0.0.0/0            0.0.0.0/0           tcp dpt:80 
    0     0 ACCEPT     all  --  eth1   *       0.0.0.0/0            0.0.0.0/0           state ESTABLISHED 

Chain OUTPUT (policy ACCEPT 2126 packets, 172K bytes)
 pkts bytes target     prot opt in     out     source               destination     
\end{Verbatim}

Выполним \textbf{iptables -L -nv -t nat} на \textbf{r2}:
\begin{Verbatim}
Chain PREROUTING (policy ACCEPT 256 packets, 36517 bytes)
 pkts bytes target     prot opt in     out     source               destination         

Chain POSTROUTING (policy ACCEPT 16 packets, 920 bytes)
 pkts bytes target     prot opt in     out     source               destination         
    0     0 MASQUERADE  all  --  *      eth1    0.0.0.0/0            0.0.0.0/0           

Chain OUTPUT (policy ACCEPT 27 packets, 1648 bytes)
 pkts bytes target     prot opt in     out     source               destination 
\end{Verbatim}

\section{Проверка трансляции}

Для демонстрации SNAT преобразований выполним \textbf{telnet 213.180.193.3 80} на хосте \textbf{ws11}.

Вывод \textbf{tcpdump -i eth0 -p tcp -tnv} на хосте на \textbf{ws11}:
\begin{Verbatim}
tcpdump: listening on eth0, link-type EN10MB (Ethernet), capture size 96 bytes
IP (tos 0x10, ttl 64, id 46644, offset 0, flags [DF], proto TCP (6), length 60)
 10.0.1.1.55843 > 213.180.193.3.80: S, cksum 0xf413 (correct), 2326482989:23264
82989(0) win 5840 <mss 1460,sackOK,timestamp 4294960665 0,nop,wscale 4>
IP (tos 0x0, ttl 54, id 0, offset 0, flags [DF], proto TCP (6), length 52) 213.180.193.3.80 > 10.0.1.1.55843: S, cksum 0x5250 (correct), 3348940585:3348940585(0) ack 2326482990 win 14100 <mss 1410,nop,nop,sackOK,nop,wscale 9>
IP (tos 0x10, ttl 64, id 46645, offset 0, flags [DF], proto TCP (6), length 40) 10.0.1.1.55843 > 213.180.193.3.80: ., cksum 0xc899 (correct), ack 1 win 365
IP (tos 0x10, ttl 64, id 46646, offset 0, flags [DF], proto TCP (6), length 46) 10.0.1.1.55843 > 213.180.193.3.80: P, cksum 0xd3a7 (correct), 1:7(6) ack 1 win 365
IP (tos 0x0, ttl 54, id 44409, offset 0, flags [DF], proto TCP (6), length 40) 213.180.193.3.80 > 10.0.1.1.55843: ., cksum 0xc9e4 (correct), ack 7 win 28
IP (tos 0x0, ttl 54, id 44410, offset 0, flags [DF], proto TCP (6), length 206) 213.180.193.3.80 > 10.0.1.1.55843: P 1:167(166) ack 7 win 28
IP (tos 0x10, ttl 64, id 46647, offset 0, flags [DF], proto TCP (6), length 40) 10.0.1.1.55843 > 213.180.193.3.80: ., cksum 0xc7aa (correct), ack 167 win 432
IP (tos 0x0, ttl 54, id 44411, offset 0, flags [DF], proto TCP (6), length 40) 213.180.193.3.80 > 10.0.1.1.55843: F, cksum 0xc93d (correct), 167:167(0) ack 7 win 28
IP (tos 0x10, ttl 64, id 46648, offset 0, flags [DF], proto TCP (6), length 40) 10.0.1.1.55843 > 213.180.193.3.80: F, cksum 0xc7a8 (correct), 7:7(0) ack 168 win 432
IP (tos 0x0, ttl 54, id 44412, offset 0, flags [DF], proto TCP (6), length 40) 213.180.193.3.80 > 10.0.1.1.55843: ., cksum 0xc93c (correct), ack 8 win 28
\end{Verbatim}

Вывод \textbf{tcpdump -i eth1 -p tcp -tnv}  на маршрутизаторе \textbf{r1}:
\begin{Verbatim}
IP (tos 0x10, ttl 63, id 46644, offset 0, flags [DF], proto TCP (6), length 60) 172.16.1.3.55843 > 213.180.193.3.80: S, cksum 0x5201 (correct), 2326482989:2326482989(0) win 5840 <mss 1460,sackOK,timestamp 4294960665 0,nop,wscale 4>
IP (tos 0x0, ttl 55, id 0, offset 0, flags [DF], proto TCP (6), length 52) 213.180.193.3.80 > 172.16.1.3.55843: S, cksum 0xb03d (correct), 3348940585:3348940585(0) ack 2326482990 win 14100 <mss 1410,nop,nop,sackOK,nop,wscale 9>
IP (tos 0x10, ttl 63, id 46645, offset 0, flags [DF], proto TCP (6), length 40) 172.16.1.3.55843 > 213.180.193.3.80: ., cksum 0x2687 (correct), ack 1 win 365
IP (tos 0x10, ttl 63, id 46646, offset 0, flags [DF], proto TCP (6), length 46) 172.16.1.3.55843 > 213.180.193.3.80: P, cksum 0x3195 (correct), 1:7(6) ack 1 win 365
IP (tos 0x0, ttl 55, id 44409, offset 0, flags [DF], proto TCP (6), length 40) 213.180.193.3.80 > 172.16.1.3.55843: ., cksum 0x27d2 (correct), ack 7 win 28
IP (tos 0x0, ttl 55, id 44410, offset 0, flags [DF], proto TCP (6), length 206) 213.180.193.3.80 > 172.16.1.3.55843: P 1:167(166) ack 7 win 28
IP (tos 0x0, ttl 55, id 44411, offset 0, flags [DF], proto TCP (6), length 40) 213.180.193.3.80 > 172.16.1.3.55843: F, cksum 0x272b (correct), 167:167(0) ack 7 win 28
IP (tos 0x10, ttl 63, id 46647, offset 0, flags [DF], proto TCP (6), length 40) 172.16.1.3.55843 > 213.180.193.3.80: ., cksum 0x2598 (correct), ack 167 win 432
IP (tos 0x10, ttl 63, id 46648, offset 0, flags [DF], proto TCP (6), length 40) 172.16.1.3.55843 > 213.180.193.3.80: F, cksum 0x2596 (correct), 7:7(0) ack 168 win 432
IP (tos 0x0, ttl 55, id 44412, offset 0, flags [DF], proto TCP (6), length 40) 213.180.193.3.80 > 172.16.1.3.55843: ., cksum 0x272a (correct), ack 8 win 28
\end{Verbatim}

Для демонстрации DNAT преобразований выполним \textbf{telnet 172.16.1.3 80}.

Вывод \textbf{tcpdump -i eth1 -p tcp -tnv}  на маршрутизаторе \textbf{r1}:
\begin{Verbatim}
IP (tos 0x10, ttl 64, id 3896, offset 0, flags [DF], proto TCP (6), length 60) 172.16.1.2.58126 > 172.16.1.3.80: S, cksum 0x5323 (correct), 3448690626:3448690626(0) win 14600 <mss 1460,sackOK,timestamp 514298 0,nop,wscale 4>
IP (tos 0x0, ttl 63, id 0, offset 0, flags [DF], proto TCP (6), length 60) 172.16.1.3.80 > 172.16.1.2.58126: S, cksum 0x6a0a (correct), 3793067127:3793067127(0) ack 3448690627 win 5792 <mss 1460,sackOK,timestamp 167136 514298,nop,wscale 4>
IP (tos 0x10, ttl 64, id 3897, offset 0, flags [DF], proto TCP (6), length 52) 172.16.1.2.58126 > 172.16.1.3.80: ., cksum 0xabdc (correct), ack 1 win 913 <nop,nop,timestamp 514304 167136>
IP (tos 0x0, ttl 63, id 0, offset 0, flags [DF], proto TCP (6), length 60) 172.16.1.3.80 > 172.16.1.2.58126: S, cksum 0x6888 (correct), 3793067127:3793067127(0) ack 3448690627 win 5792 <mss 1460,sackOK,timestamp 167516 514304,nop,wscale 4>
IP (tos 0x10, ttl 64, id 3898, offset 0, flags [DF], proto TCP (6), length 52) 172.16.1.2.58126 > 172.16.1.3.80: ., cksum 0xa829 (correct), ack 1 win 913 <nop,nop,timestamp 515251 167136>
IP (tos 0x10, ttl 64, id 3899, offset 0, flags [DF], proto TCP (6), length 58) 172.16.1.2.58126 > 172.16.1.3.80: P, cksum 0xb01b (correct), 1:7(6) ack 1 win 913 <nop,nop,timestamp 516047 167136>
IP (tos 0x0, ttl 63, id 5090, offset 0, flags [DF], proto TCP (6), length 52) 172.16.1.3.80 > 172.16.1.2.58126: ., cksum 0xa474 (correct), ack 7 win 362 <nop,nop,timestamp 167834 516047>
IP (tos 0x0, ttl 63, id 5091, offset 0, flags [DF], proto TCP (6), length 342) 172.16.1.3.80 > 172.16.1.2.58126: P 1:291(290) ack 7 win 362 <nop,nop,timestamp 167834 516047>
IP (tos 0x10, ttl 64, id 3900, offset 0, flags [DF], proto TCP (6), length 52) 172.16.1.2.58126 > 172.16.1.3.80: ., cksum 0xa0e4 (correct), ack 291 win 980 <nop,nop,timestamp 516051 167834>
IP (tos 0x0, ttl 63, id 5092, offset 0, flags [DF], proto TCP (6), length 52) 172.16.1.3.80 > 172.16.1.2.58126: F, cksum 0xa34d (correct), 291:291(0) ack 7 win 362 <nop,nop,timestamp 167834 516051>
IP (tos 0x10, ttl 64, id 3901, offset 0, flags [DF], proto TCP (6), length 52) 172.16.1.2.58126 > 172.16.1.3.80: F, cksum 0xa0e2 (correct), 7:7(0) ack 292 win 980 <nop,nop,timestamp 516051 167834>
IP (tos 0x0, ttl 63, id 5093, offset 0, flags [DF], proto TCP (6), length 52) 172.16.1.3.80 > 172.16.1.2.58126: ., cksum 0xa34c (correct), ack 8 win 362 <nop,nop,timestamp 167834 516051>
\end{Verbatim}

Вывод \textbf{tcpdump -i eth0 -p tcp -tnv} на хосте \textbf{s11}:
\begin{Verbatim}
IP (tos 0x10, ttl 63, id 3896, offset 0, flags [DF], proto TCP (6), length 60) 172.16.1.2.58126 > 10.0.4.20.80: S, cksum 0xf222 (correct), 3448690626:3448690626(0) win 14600 <mss 1460,sackOK,timestamp 514298 0,nop,wscale 4>
IP (tos 0x0, ttl 64, id 0, offset 0, flags [DF], proto TCP (6), length 60) 10.0.4.20.80 > 172.16.1.2.58126: S, cksum 0x090a (correct), 3793067127:3793067127(0) ack 3448690627 win 5792 <mss 1460,sackOK,timestamp 167136 514298,nop,wscale 4>
IP (tos 0x10, ttl 63, id 3897, offset 0, flags [DF], proto TCP (6), length 52) 172.16.1.2.58126 > 10.0.4.20.80: ., cksum 0x4adc (correct), ack 1 win 913 <nop,nop,timestamp 514304 167136>
IP (tos 0x0, ttl 64, id 0, offset 0, flags [DF], proto TCP (6), length 60) 10.0.4.20.80 > 172.16.1.2.58126: S, cksum 0x0788 (correct), 3793067127:3793067127(0) ack 3448690627 win 5792 <mss 1460,sackOK,timestamp 167516 514304,nop,wscale 4>
IP (tos 0x10, ttl 63, id 3898, offset 0, flags [DF], proto TCP (6), length 52) 172.16.1.2.58126 > 10.0.4.20.80: ., cksum 0x4729 (correct), ack 1 win 913 <nop,nop,timestamp 515251 167136>
IP (tos 0x10, ttl 63, id 3899, offset 0, flags [DF], proto TCP (6), length 58) 172.16.1.2.58126 > 10.0.4.20.80: P, cksum 0x4f1b (correct), 1:7(6) ack 1 win 913 <nop,nop,timestamp 516047 167136>
IP (tos 0x0, ttl 64, id 5090, offset 0, flags [DF], proto TCP (6), length 52) 10.0.4.20.80 > 172.16.1.2.58126: ., cksum 0x4374 (correct), ack 7 win 362 <nop,nop,timestamp 167834 516047>
IP (tos 0x0, ttl 64, id 5091, offset 0, flags [DF], proto TCP (6), length 342) 10.0.4.20.80 > 172.16.1.2.58126: P 1:291(290) ack 7 win 362 <nop,nop,timestamp 167834 516047>
IP (tos 0x10, ttl 63, id 3900, offset 0, flags [DF], proto TCP (6), length 52) 172.16.1.2.58126 > 10.0.4.20.80: ., cksum 0x3fe4 (correct), ack 291 win 980 <nop,nop,timestamp 516051 167834>
IP (tos 0x0, ttl 64, id 5092, offset 0, flags [DF], proto TCP (6), length 52) 10.0.4.20.80 > 172.16.1.2.58126: F, cksum 0x424d (correct), 291:291(0) ack 7 win 362 <nop,nop,timestamp 167834 516051>
IP (tos 0x10, ttl 63, id 3901, offset 0, flags [DF], proto TCP (6), length 52) 172.16.1.2.58126 > 10.0.4.20.80: F, cksum 0x3fe2 (correct), 7:7(0) ack 292 win 980 <nop,nop,timestamp 516051 167834>
IP (tos 0x0, ttl 64, id 5093, offset 0, flags [DF], proto TCP (6), length 52) 10.0.4.20.80 > 172.16.1.2.58126: ., cksum 0x424c (correct), ack 8 win 362 <nop,nop,timestamp 167834 516051>
\end{Verbatim}


\section{Проверка правил фильтрации}

Проверим правильность найтройки правил фильтрации на маршрутизаторах
с использованием  telnet и traceroute.

С хоста \textbf{ws11} должны быть доступны любые соединения по vpn и соединения на
порте 80, кроме сервисов vkontakte.ru и vk.com.
Выполним с хоста \textbf{ws11} \textbf{telnet 87.240.156.161 80} для проверки запрещения подключения
к сервису vkontakte.ru на порту 80:
\begin{Verbatim}
ws11:~# telnet 87.240.156.161 80
Trying 87.240.156.161...
\end{Verbatim}

Вывод \textbf{tcpdump -i any -tnv -p tcp} на \textbf{r1}:
\begin{Verbatim}
IP (tos 0x10, ttl 64, id 7648, offset 0, flags [DF], proto TCP (6), length 60) 10.0.1.1.52593 > 87.240.156.161.80: S, cksum 0x049e (correct), 19178907:19178907(0) win 5840 <mss 1460,sackOK,timestamp 440443 0,nop,wscale 4>
\end{Verbatim}

Выполним с хоста \textbf{ws11} \textbf{telnet 213.180.193.3 80},  чтобы проверить, что
происходит подключение к другим сервисам, например, ya.ru:
\begin{Verbatim}
ws11:~# telnet 213.180.193.3 80
Trying 213.180.193.3...
Connected to 213.180.193.3.
Escape character is '^]'.
\end{Verbatim}

Вывод \textbf{tcpdump -i any -tnv -p tcp} на \textbf{r1}:
\begin{Verbatim}
IP (tos 0x10, ttl 64, id 49528, offset 0, flags [DF], proto TCP (6), length 60) 10.0.1.1.58316 > 213.180.193.3.80: S, cksum 0xc5aa (correct), 472909138:472909138(0) win 5840 <mss 1460,sackOK,timestamp 443434 0,nop,wscale 4>
IP (tos 0x10, ttl 63, id 49528, offset 0, flags [DF], proto TCP (6), length 60) 172.16.1.3.58316 > 213.180.193.3.80: S, cksum 0x2398 (correct), 472909138:472909138(0) win 5840 <mss 1460,sackOK,timestamp 443434 0,nop,wscale 4>
IP (tos 0x0, ttl 52, id 0, offset 0, flags [DF], proto TCP (6), length 52) 213.180.193.3.80 > 172.16.1.3.58316: S, cksum 0x81de (correct), 669402349:669402349(0) ack 472909139 win 14100 <mss 1410,nop,nop,sackOK,nop,wscale 9>
IP (tos 0x0, ttl 51, id 0, offset 0, flags [DF], proto TCP (6), length 52) 213.180.193.3.80 > 10.0.1.1.58316: S, cksum 0x23f1 (correct), 669402349:669402349(0) ack 472909139 win 14100 <mss 1410,nop,nop,sackOK,nop,wscale 9>
IP (tos 0x10, ttl 64, id 49529, offset 0, flags [DF], proto TCP (6), length 40) 10.0.1.1.58316 > 213.180.193.3.80: ., cksum 0x9a3a (correct), ack 1 win 365
IP (tos 0x10, ttl 63, id 49529, offset 0, flags [DF], proto TCP (6), length 40) 172.16.1.3.58316 > 213.180.193.3.80: ., cksum 0xf827 (correct), ack 1 win 365
IP (tos 0x10, ttl 64, id 49530, offset 0, flags [DF], proto TCP (6), length 40) 10.0.1.1.58316 > 213.180.193.3.80: F, cksum 0x9a39 (correct), 1:1(0) ack 1 win 365
IP (tos 0x10, ttl 63, id 49530, offset 0, flags [DF], proto TCP (6), length 40) 172.16.1.3.58316 > 213.180.193.3.80: F, cksum 0xf826 (correct), 1:1(0) ack 1 win 365
IP (tos 0x0, ttl 52, id 16650, offset 0, flags [DF], proto TCP (6), length 40) 213.180.193.3.80 > 172.16.1.3.58316: F, cksum 0xf976 (correct), 1:1(0) ack 2 win 28
IP (tos 0x0, ttl 51, id 16650, offset 0, flags [DF], proto TCP (6), length 40) 213.180.193.3.80 > 10.0.1.1.58316: F, cksum 0x9b89 (correct), 1:1(0) ack 2 win 28
IP (tos 0x10, ttl 64, id 49531, offset 0, flags [DF], proto TCP (6), length 40) 10.0.1.1.58316 > 213.180.193.3.80: ., cksum 0x9a38 (correct), ack 2 win 365
IP (tos 0x10, ttl 63, id 49531, offset 0, flags [DF], proto TCP (6), length 40) 172.16.1.3.58316 > 213.180.193.3.80: ., cksum 0xf825 (correct), ack 2 win 365
\end{Verbatim}

Выполним с хоста \textbf{ws11} \textbf{telnet 217.69.139.90 143},  чтобы проверить,
что не происходит подключение к сервисам, например, imap.mail.ru на порту 143:
\begin{Verbatim}
ws11:~# telnet 217.69.139.90 143
Trying 217.69.139.90...
\end{Verbatim}

Вывод \textbf{tcpdump -i any -tnv -p tcp} на \textbf{r1}:
\begin{Verbatim}
IP (tos 0x10, ttl 64, id 31584, offset 0, flags [DF], proto TCP (6), length 60) 10.0.1.1.40282 > 217.69.139.90.143: S, cksum 0x65f5 (correct), 1092135162:1092135162(0) win 5840 <mss 1460,sackOK,timestamp 447386 0,nop,wscale 4>
\end{Verbatim}

Выполним с хоста \textbf{ws11} \textbf{traceroute -n 10.102.0.2},  чтобы проверить,
что разрешен vpn:
\begin{Verbatim}
ws11:~# traceroute -n 10.102.0.2
traceroute to 10.102.0.2 (10.102.0.2), 64 hops max, 40 byte packets
 1 ^[[6~ 10.0.0.1  1 ms  0 ms  0 ms
 2  10.200.1.2  1 ms  1 ms  1 ms
 3  10.102.0.2  1 ms  1 ms  1 ms
\end{Verbatim}


С хоста \textbf{ws12} должны быть доступны любые соединения по vpn и
соедиения на порте 80, 443 только для mail.ru и rambler.ru. 

Выполним с хоста \textbf{ws12} \textbf{telnet 217.69.139.199 80} для проверки
разрешения подключения к mail.ru на порту 80:
\begin{Verbatim}
ws12:~# telnet 217.69.139.199 80
Trying 217.69.139.199...
Connected to 217.69.139.199.
Escape character is '^]'.
\end{Verbatim}

Вывод \textbf{tcpdump -i any -tnv -p tcp} на \textbf{r1}:
\begin{Verbatim}
IP (tos 0x10, ttl 64, id 37238, offset 0, flags [DF], proto TCP (6), length 60) 10.0.2.1.57399 > 217.69.139.199.80: S, cksum 0x566e (correct), 1538412372:1538412372(0) win 5840 <mss 1460,sackOK,timestamp 450082 0,nop,wscale 4>
IP (tos 0x10, ttl 63, id 37238, offset 0, flags [DF], proto TCP (6), length 60) 172.16.1.3.57399 > 217.69.139.199.80: S, cksum 0xb55b (correct), 1538412372:1538412372(0) win 5840 <mss 1460,sackOK,timestamp 450082 0,nop,wscale 4>
IP (tos 0x0, ttl 56, id 0, offset 0, flags [DF], proto TCP (6), length 56) 217.69.139.199.80 > 172.16.1.3.57399: S, cksum 0xa488 (correct), 958865643:958865643(0) ack 1538412373 win 5592 <mss 1410,sackOK,timestamp 509717891 450082>
IP (tos 0x0, ttl 55, id 0, offset 0, flags [DF], proto TCP (6), length 56) 217.69.139.199.80 > 10.0.2.1.57399: S, cksum 0x459b (correct), 958865643:958865643(0) ack 1538412373 win 5592 <mss 1410,sackOK,timestamp 509717891 450082>
IP (tos 0x10, ttl 64, id 37239, offset 0, flags [DF], proto TCP (6), length 52) 10.0.2.1.57399 > 217.69.139.199.80: ., cksum 0x5f2e (correct), ack 1 win 5840 <nop,nop,timestamp 450083 509717891>
IP (tos 0x10, ttl 63, id 37239, offset 0, flags [DF], proto TCP (6), length 52) 172.16.1.3.57399 > 217.69.139.199.80: ., cksum 0xbe1b (correct), ack 1 win 5840 <nop,nop,timestamp 450083 509717891>
IP (tos 0x10, ttl 64, id 37240, offset 0, flags [DF], proto TCP (6), length 52) 10.0.2.1.57399 > 217.69.139.199.80: F, cksum 0x5c8b (correct), 1:1(0) ack 1 win 5840 <nop,nop,timestamp 450757 509717891>
IP (tos 0x10, ttl 63, id 37240, offset 0, flags [DF], proto TCP (6), length 52) 172.16.1.3.57399 > 217.69.139.199.80: F, cksum 0xbb78 (correct), 1:1(0) ack 1 win 5840 <nop,nop,timestamp 450757 509717891>
IP (tos 0x0, ttl 56, id 36198, offset 0, flags [DF], proto TCP (6), length 52) 217.69.139.199.80 > 172.16.1.3.57399: F, cksum 0xa217 (correct), 1:1(0) ack 2 win 5592 <nop,nop,timestamp 509724635 450757>
IP (tos 0x0, ttl 55, id 36198, offset 0, flags [DF], proto TCP (6), length 52) 217.69.139.199.80 > 10.0.2.1.57399: F, cksum 0x432a (correct), 1:1(0) ack 2 win 5592 <nop,nop,timestamp 509724635 450757>
IP (tos 0x10, ttl 64, id 37241, offset 0, flags [DF], proto TCP (6), length 52) 10.0.2.1.57399 > 217.69.139.199.80: ., cksum 0x4229 (correct), ack 2 win 5840 <nop,nop,timestamp 450766 509724635>
IP (tos 0x10, ttl 63, id 37241, offset 0, flags [DF], proto TCP (6), length 52) 172.16.1.3.57399 > 217.69.139.199.80: ., cksum 0xa116 (correct), ack 2 win 5840 <nop,nop,timestamp 450766 509724635>
\end{Verbatim}

Выполним с хоста \textbf{ws12} \textbf{telnet 217.69.139.199 443} для проверки
разрешения подключения к mail.ru на порту 443:
\begin{Verbatim}
ws12:~# telnet 217.69.139.199 443
Trying 217.69.139.199...
Connected to 217.69.139.199.
Escape character is '^]'.
\end{Verbatim}

Вывод \textbf{tcpdump -i any -tnv -p tcp} на \textbf{r1}:
\begin{Verbatim}
IP (tos 0x10, ttl 64, id 46442, offset 0, flags [DF], proto TCP (6), length 60) 10.0.2.1.33658 > 217.69.139.199.443: S, cksum 0x76c3 (correct), 2109299145:2109299145(0) win 5840 <mss 1460,sackOK,timestamp 453795 0,nop,wscale 4>
IP (tos 0x10, ttl 63, id 46442, offset 0, flags [DF], proto TCP (6), length 60) 172.16.1.3.33658 > 217.69.139.199.443: S, cksum 0xd5b0 (correct), 2109299145:2109299145(0) win 5840 <mss 1460,sackOK,timestamp 453795 0,nop,wscale 4>
IP (tos 0x0, ttl 56, id 0, offset 0, flags [DF], proto TCP (6), length 56) 217.69.139.199.443 > 172.16.1.3.33658: S, cksum 0x46f0 (correct), 1506870879:1506870879(0) ack 2109299146 win 5592 <mss 1410,sackOK,timestamp 1759107802 453795>
IP (tos 0x0, ttl 55, id 0, offset 0, flags [DF], proto TCP (6), length 56) 217.69.139.199.443 > 10.0.2.1.33658: S, cksum 0xe802 (correct), 1506870879:1506870879(0) ack 2109299146 win 5592 <mss 1410,sackOK,timestamp 1759107802 453795>
IP (tos 0x10, ttl 64, id 46443, offset 0, flags [DF], proto TCP (6), length 52) 10.0.2.1.33658 > 217.69.139.199.443: ., cksum 0x0197 (correct), ack 1 win 5840 <nop,nop,timestamp 453795 1759107802>
IP (tos 0x10, ttl 63, id 46443, offset 0, flags [DF], proto TCP (6), length 52) 172.16.1.3.33658 > 217.69.139.199.443: ., cksum 0x6084 (correct), ack 1 win 5840 <nop,nop,timestamp 453795 1759107802>
IP (tos 0x10, ttl 64, id 46444, offset 0, flags [DF], proto TCP (6), length 52) 10.0.2.1.33658 > 217.69.139.199.443: F, cksum 0x00db (correct), 1:1(0) ack 1 win 5840 <nop,nop,timestamp 453982 1759107802>
IP (tos 0x10, ttl 63, id 46444, offset 0, flags [DF], proto TCP (6), length 52) 172.16.1.3.33658 > 217.69.139.199.443: F, cksum 0x5fc8 (correct), 1:1(0) ack 1 win 5840 <nop,nop,timestamp 453982 1759107802>
IP (tos 0x0, ttl 56, id 56510, offset 0, flags [DF], proto TCP (6), length 52) 217.69.139.199.443 > 172.16.1.3.33658: F, cksum 0x599c (correct), 1:1(0) ack 2 win 5592 <nop,nop,timestamp 1759109629 453982>
IP (tos 0x0, ttl 55, id 56510, offset 0, flags [DF], proto TCP (6), length 52) 217.69.139.199.443 > 10.0.2.1.33658: F, cksum 0xfaae (correct), 1:1(0) ack 2 win 5592 <nop,nop,timestamp 1759109629 453982>
IP (tos 0x10, ttl 64, id 46445, offset 0, flags [DF], proto TCP (6), length 52) 10.0.2.1.33658 > 217.69.139.199.443: ., cksum 0xf9b6 (correct), ack 2 win 5840 <nop,nop,timestamp 453982 1759109629>
IP (tos 0x10, ttl 63, id 46445, offset 0, flags [DF], proto TCP (6), length 52) 172.16.1.3.33658 > 217.69.139.199.443: ., cksum 0x58a4 (correct), ack 2 win 5840 <nop,nop,timestamp 453982 1759109629>
\end{Verbatim}

Выполним с хоста \textbf{ws12} \textbf{telnet 81.19.70.3 80} для проверки
разрешения подключения к rambler.ru на порту 80:
\begin{Verbatim}
ws12:~# telnet 81.19.70.3 80
Trying 81.19.70.3...
Connected to 81.19.70.3.
Escape character is '^]'.
\end{Verbatim}

Вывод \textbf{tcpdump -i any -tnv -p tcp} на \textbf{r1}:
\begin{Verbatim}
IP (tos 0x10, ttl 64, id 7740, offset 0, flags [DF], proto TCP (6), length 60) 10.0.2.1.33000 > 81.19.70.3.80: S, cksum 0x88f4 (correct), 2720523774:2720523774(0) win 5840 <mss 1460,sackOK,timestamp 457666 0,nop,wscale 4>
IP (tos 0x10, ttl 63, id 7740, offset 0, flags [DF], proto TCP (6), length 60) 172.16.1.3.33000 > 81.19.70.3.80: S, cksum 0xe7e1 (correct), 2720523774:2720523774(0) win 5840 <mss 1460,sackOK,timestamp 457666 0,nop,wscale 4>
IP (tos 0x0, ttl 55, id 28336, offset 0, flags [DF], proto TCP (6), length 48) 81.19.70.3.80 > 172.16.1.3.33000: S, cksum 0xe425 (correct), 1940111037:1940111037(0) ack 2720523775 win 8192 <mss 1460,sackOK,eol>
IP (tos 0x0, ttl 54, id 28336, offset 0, flags [DF], proto TCP (6), length 48) 81.19.70.3.80 > 10.0.2.1.33000: S, cksum 0x8538 (correct), 1940111037:1940111037(0) ack 2720523775 win 8192 <mss 1460,sackOK,eol>
IP (tos 0x10, ttl 64, id 7741, offset 0, flags [DF], proto TCP (6), length 40) 10.0.2.1.33000 > 81.19.70.3.80: ., cksum 0xba2b (correct), ack 1 win 5840
IP (tos 0x10, ttl 63, id 7741, offset 0, flags [DF], proto TCP (6), length 40) 172.16.1.3.33000 > 81.19.70.3.80: ., cksum 0x1919 (correct), ack 1 win 5840
IP (tos 0x10, ttl 64, id 7742, offset 0, flags [DF], proto TCP (6), length 40) 10.0.2.1.33000 > 81.19.70.3.80: F, cksum 0xba2a (correct), 1:1(0) ack 1 win 5840
IP (tos 0x10, ttl 63, id 7742, offset 0, flags [DF], proto TCP (6), length 40) 172.16.1.3.33000 > 81.19.70.3.80: F, cksum 0x1918 (correct), 1:1(0) ack 1 win 5840
IP (tos 0x0, ttl 55, id 31838, offset 0, flags [DF], proto TCP (6), length 40) 81.19.70.3.80 > 172.16.1.3.33000: ., cksum 0x0db0 (correct), ack 2 win 8760
IP (tos 0x0, ttl 54, id 31838, offset 0, flags [DF], proto TCP (6), length 40) 81.19.70.3.80 > 10.0.2.1.33000: ., cksum 0xaec2 (correct), ack 2 win 8760
IP (tos 0x0, ttl 55, id 31839, offset 0, flags [DF], proto TCP (6), length 40) 81.19.70.3.80 > 172.16.1.3.33000: F, cksum 0x0daf (correct), 1:1(0) ack 2 win 8760
IP (tos 0x0, ttl 54, id 31839, offset 0, flags [DF], proto TCP (6), length 40) 81.19.70.3.80 > 10.0.2.1.33000: F, cksum 0xaec1 (correct), 1:1(0) ack 2 win 8760
IP (tos 0x0, ttl 64, id 0, offset 0, flags [DF], proto TCP (6), length 40) 10.0.2.1.33000 > 81.19.70.3.80: ., cksum 0xba29 (correct), ack 2 win 5840
IP (tos 0x0, ttl 63, id 0, offset 0, flags [DF], proto TCP (6), length 40) 172.16.1.3.33000 > 81.19.70.3.80: ., cksum 0x1917 (correct), ack 2 win 5840
\end{Verbatim}

Выполним с хоста \textbf{ws12} \textbf{telnet 81.19.70.3 443} для проверки разрешения
подключения к rambler.ru на порту 443:
\begin{Verbatim}
ws12:~# telnet 81.19.70.3 443
Trying 81.19.70.3...
Connected to 81.19.70.3.
Escape character is '^]'.
\end{Verbatim}

Вывод \textbf{tcpdump -i any -tnv -p tcp} на \textbf{r1}:
\begin{Verbatim}
IP (tos 0x10, ttl 64, id 4916, offset 0, flags [DF], proto TCP (6), length 60) 10.0.2.1.55741 > 81.19.70.3.443: S, cksum 0xa7c2 (correct), 3869576064:3869576064(0) win 5840 <mss 1460,sackOK,timestamp 465076 0,nop,wscale 4>
IP (tos 0x10, ttl 63, id 4916, offset 0, flags [DF], proto TCP (6), length 60) 172.16.1.3.55741 > 81.19.70.3.443: S, cksum 0x06b0 (correct), 3869576064:3869576064(0) win 5840 <mss 1460,sackOK,timestamp 465076 0,nop,wscale 4>
IP (tos 0x10, ttl 64, id 4917, offset 0, flags [DF], proto TCP (6), length 60) 10.0.2.1.55741 > 81.19.70.3.443: S, cksum 0xa696 (correct), 3869576064:3869576064(0) win 5840 <mss 1460,sackOK,timestamp 465376 0,nop,wscale 4>
IP (tos 0x10, ttl 63, id 4917, offset 0, flags [DF], proto TCP (6), length 60) 172.16.1.3.55741 > 81.19.70.3.443: S, cksum 0x0584 (correct), 3869576064:3869576064(0) win 5840 <mss 1460,sackOK,timestamp 465376 0,nop,wscale 4>
\end{Verbatim}

Выполним с хоста \textbf{ws12} \textbf{telnet 213.180.193.3 80},  чтобы проверить, что
не происходит подключение к другим сервисам, например, ya.ru:
\begin{Verbatim}
ws12:~# telnet 213.180.193.3 80
Trying 213.180.193.3...
\end{Verbatim}

Вывод \textbf{tcpdump -i any -tnv -p tcp} на \textbf{r1}:
\begin{Verbatim}
IP (tos 0x10, ttl 64, id 18013, offset 0, flags [DF], proto TCP (6), length 60) 10.0.2.1.55017 > 213.180.193.3.80: S, cksum 0x13b9 (correct), 215969339:215969339(0) win 5840 <mss 1460,sackOK,timestamp 469094 0,nop,wscale 4>
IP (tos 0x10, ttl 64, id 18014, offset 0, flags [DF], proto TCP (6), length 60) 10.0.2.1.55017 > 213.180.193.3.80: S, cksum 0x128d (correct), 215969339:215969339(0) win 5840 <mss 1460,sackOK,timestamp 469394 0,nop,wscale 4>
\end{Verbatim}

Выполним с хоста \textbf{ws12} \textbf{telnet 217.69.139.90 143},  чтобы проверить,
что не происходит подключение к сервисам, например, imap.mail.ru на порту 143:
\begin{Verbatim}
ws12:~# telnet 217.69.139.90 143
Trying 217.69.139.90...
\end{Verbatim}

Вывод \textbf{tcpdump -i any -tnv -p tcp} на \textbf{r1}:
\begin{Verbatim}
IP (tos 0x10, ttl 64, id 7352, offset 0, flags [DF], proto TCP (6), length 60) 10.0.2.1.55391 > 217.69.139.90.143: S, cksum 0xfb6f (correct), 1324243155:1324243155(0) win 5840 <mss 1460,sackOK,timestamp 476267 0,nop,wscale 4>
\end{Verbatim}

Выполним с хоста \textbf{ws12} \textbf{traceroute -n 10.102.0.2},  чтобы проверить,
что разрешен vpn:
\begin{Verbatim}
ws12:~# traceroute -n 10.102.0.2
traceroute to 10.102.0.2 (10.102.0.2), 64 hops max, 40 byte packets
 1  10.0.0.1  8 ms  0 ms  0 ms
 2  10.200.1.2  1 ms  2 ms  4 ms
 3  10.102.0.2  2 ms  1 ms  1 ms
\end{Verbatim}


С хоста \textbf{ws13} должны быть доступны любые соединения по vpn и соедиения
на порте 80, 110, 443.

Выполним с хоста \textbf{ws13} \textbf{telnet 213.180.193.3 80},  чтобы проверить, что
происходит подключение к сервисам, например, ya.ru на порту 80:
\begin{Verbatim}
ws13:~# telnet 213.180.193.3 80
Trying 213.180.193.3...
Connected to 213.180.193.3.
Escape character is '^]'.
\end{Verbatim}

Вывод \textbf{tcpdump -i any -tnv -p tcp} на \textbf{r1}:
\begin{Verbatim}
IP (tos 0x10, ttl 64, id 45525, offset 0, flags [DF], proto TCP (6), length 60) 10.0.3.1.59365 > 213.180.193.3.80: S, cksum 0x06b6 (correct), 1818353961:1818353961(0) win 5840 <mss 1460,sackOK,timestamp 479484 0,nop,wscale 4>
IP (tos 0x10, ttl 63, id 45525, offset 0, flags [DF], proto TCP (6), length 60) 172.16.1.3.59365 > 213.180.193.3.80: S, cksum 0x66a3 (correct), 1818353961:1818353961(0) win 5840 <mss 1460,sackOK,timestamp 479484 0,nop,wscale 4>
IP (tos 0x0, ttl 53, id 0, offset 0, flags [DF], proto TCP (6), length 52) 213.180.193.3.80 > 172.16.1.3.59365: S, cksum 0x739f (correct), 3298920014:3298920014(0) ack 1818353962 win 14100 <mss 1410,nop,nop,sackOK,nop,wscale 9>
IP (tos 0x0, ttl 52, id 0, offset 0, flags [DF], proto TCP (6), length 52) 213.180.193.3.80 > 10.0.3.1.59365: S, cksum 0x13b2 (correct), 3298920014:3298920014(0) ack 1818353962 win 14100 <mss 1410,nop,nop,sackOK,nop,wscale 9>
IP (tos 0x10, ttl 64, id 45526, offset 0, flags [DF], proto TCP (6), length 40) 10.0.3.1.59365 > 213.180.193.3.80: ., cksum 0x89fb (correct), ack 1 win 365
IP (tos 0x10, ttl 63, id 45526, offset 0, flags [DF], proto TCP (6), length 40) 172.16.1.3.59365 > 213.180.193.3.80: ., cksum 0xe9e8 (correct), ack 1 win 365
IP (tos 0x10, ttl 64, id 45527, offset 0, flags [DF], proto TCP (6), length 40) 10.0.3.1.59365 > 213.180.193.3.80: F, cksum 0x89fa (correct), 1:1(0) ack 1 win 365
IP (tos 0x10, ttl 63, id 45527, offset 0, flags [DF], proto TCP (6), length 40) 172.16.1.3.59365 > 213.180.193.3.80: F, cksum 0xe9e7 (correct), 1:1(0) ack 1 win 365
IP (tos 0x0, ttl 53, id 62406, offset 0, flags [DF], proto TCP (6), length 40) 213.180.193.3.80 > 172.16.1.3.59365: F, cksum 0xeb37 (correct), 1:1(0) ack 2 win 28
IP (tos 0x0, ttl 52, id 62406, offset 0, flags [DF], proto TCP (6), length 40) 213.180.193.3.80 > 10.0.3.1.59365: F, cksum 0x8b4a (correct), 1:1(0) ack 2 win 28
IP (tos 0x10, ttl 64, id 45528, offset 0, flags [DF], proto TCP (6), length 40) 10.0.3.1.59365 > 213.180.193.3.80: ., cksum 0x89f9 (correct), ack 2 win 365
IP (tos 0x10, ttl 63, id 45528, offset 0, flags [DF], proto TCP (6), length 40) 172.16.1.3.59365 > 213.180.193.3.80: ., cksum 0xe9e6 (correct), ack 2 win 365
\end{Verbatim}

Выполним с него \textbf{telnet 213.180.204.11 443},  чтобы проверить, что
происходит подключение к сервисам, например, yandex.ru на порту 443:
\begin{Verbatim}
ws13:~# telnet 213.180.204.11 443
Trying 213.180.204.11...
Connected to 213.180.204.11.
Escape character is '^]'.
\end{Verbatim}

Вывод \textbf{tcpdump -i any -tnv -p tcp} на \textbf{r1}:
\begin{Verbatim}
IP (tos 0x10, ttl 64, id 44435, offset 0, flags [DF], proto TCP (6), length 60) 10.0.3.1.36318 > 213.180.204.11.443: S, cksum 0x2594 (correct), 2674509875:2674509875(0) win 5840 <mss 1460,sackOK,timestamp 484768 0,nop,wscale 4>
IP (tos 0x10, ttl 63, id 44435, offset 0, flags [DF], proto TCP (6), length 60) 172.16.1.3.36318 > 213.180.204.11.443: S, cksum 0x8581 (correct), 2674509875:2674509875(0) win 5840 <mss 1460,sackOK,timestamp 484768 0,nop,wscale 4>
IP (tos 0x0, ttl 55, id 0, offset 0, flags [DF], proto TCP (6), length 60) 213.180.204.11.443 > 172.16.1.3.36318: S, cksum 0xfce0 (correct), 3783798451:3783798451(0) ack 2674509876 win 17796 <mss 8910,sackOK,timestamp 539888723 484768,nop,wscale 8>
IP (tos 0x0, ttl 54, id 0, offset 0, flags [DF], proto TCP (6), length 60) 213.180.204.11.443 > 10.0.3.1.36318: S, cksum 0x9cf3 (correct), 3783798451:3783798451(0) ack 2674509876 win 17796 <mss 8910,sackOK,timestamp 539888723 484768,nop,wscale 8>
IP (tos 0x10, ttl 64, id 44436, offset 0, flags [DF], proto TCP (6), length 52) 10.0.3.1.36318 > 213.180.204.11.443: ., cksum 0x2cf2 (correct), ack 1 win 365 <nop,nop,timestamp 484768 539888723>
IP (tos 0x10, ttl 63, id 44436, offset 0, flags [DF], proto TCP (6), length 52) 172.16.1.3.36318 > 213.180.204.11.443: ., cksum 0x8cdf (correct), ack 1 win 365 <nop,nop,timestamp 484768 539888723>
IP (tos 0x10, ttl 64, id 44437, offset 0, flags [DF], proto TCP (6), length 52) 10.0.3.1.36318 > 213.180.204.11.443: F, cksum 0x2ba6 (correct), 1:1(0) ack 1 win 365 <nop,nop,timestamp 485099 539888723>
IP (tos 0x10, ttl 63, id 44437, offset 0, flags [DF], proto TCP (6), length 52) 172.16.1.3.36318 > 213.180.204.11.443: F, cksum 0x8b93 (correct), 1:1(0) ack 1 win 365 <nop,nop,timestamp 485099 539888723>
IP (tos 0x0, ttl 55, id 8500, offset 0, flags [DF], proto TCP (6), length 52) 213.180.204.11.443 > 172.16.1.3.36318: F, cksum 0x898c (correct), 1:1(0) ack 2 win 70 <nop,nop,timestamp 539889536 485099>
IP (tos 0x0, ttl 54, id 8500, offset 0, flags [DF], proto TCP (6), length 52) 213.180.204.11.443 > 10.0.3.1.36318: F, cksum 0x299f (correct), 1:1(0) ack 2 win 70 <nop,nop,timestamp 539889536 485099>
IP (tos 0x10, ttl 64, id 44438, offset 0, flags [DF], proto TCP (6), length 52) 10.0.3.1.36318 > 213.180.204.11.443: ., cksum 0x2878 (correct), ack 2 win 365 <nop,nop,timestamp 485099 539889536>
IP (tos 0x10, ttl 63, id 44438, offset 0, flags [DF], proto TCP (6), length 52) 172.16.1.3.36318 > 213.180.204.11.443: ., cksum 0x8865 (correct), ack 2 win 365 <nop,nop,timestamp 485099 539889536>
\end{Verbatim}

Выполним с него \textbf{telnet 217.69.139.74 110},  чтобы проверить, что
происходит подключение к сервисам, например, pop.mail.ru на порту 110:
\begin{Verbatim}
ws13:~# telnet 217.69.139.74 110
Trying 217.69.139.74...
Connected to 217.69.139.74.
Escape character is '^]'.
+OK
\end{Verbatim}

Вывод \textbf{tcpdump -i any -tnv -p tcp} на \textbf{r1}:
\begin{Verbatim}
IP (tos 0x10, ttl 64, id 35729, offset 0, flags [DF], proto TCP (6), length 60) 10.0.3.1.40904 > 217.69.139.74.110: S, cksum 0xe28c (correct), 3504992559:3504992559(0) win 5840 <mss 1460,sackOK,timestamp 490174 0,nop,wscale 4>
IP (tos 0x10, ttl 63, id 35729, offset 0, flags [DF], proto TCP (6), length 60) 172.16.1.3.40904 > 217.69.139.74.110: S, cksum 0x427a (correct), 3504992559:3504992559(0) win 5840 <mss 1460,sackOK,timestamp 490174 0,nop,wscale 4>
IP (tos 0x0, ttl 55, id 0, offset 0, flags [DF], proto TCP (6), length 56) 217.69.139.74.110 > 172.16.1.3.40904: S, cksum 0xeec5 (correct), 3565825369:3565825369(0) ack 3504992560 win 5592 <mss 1410,sackOK,timestamp 891498961 490174>
IP (tos 0x0, ttl 54, id 0, offset 0, flags [DF], proto TCP (6), length 56) 217.69.139.74.110 > 10.0.3.1.40904: S, cksum 0x8ed8 (correct), 3565825369:3565825369(0) ack 3504992560 win 5592 <mss 1410,sackOK,timestamp 891498961 490174>
IP (tos 0x10, ttl 64, id 35730, offset 0, flags [DF], proto TCP (6), length 52) 10.0.3.1.40904 > 217.69.139.74.110: ., cksum 0xa866 (correct), ack 1 win 5840 <nop,nop,timestamp 490180 891498961>
IP (tos 0x10, ttl 63, id 35730, offset 0, flags [DF], proto TCP (6), length 52) 172.16.1.3.40904 > 217.69.139.74.110: ., cksum 0x0854 (correct), ack 1 win 5840 <nop,nop,timestamp 490180 891498961>
IP (tos 0x0, ttl 55, id 7311, offset 0, flags [DF], proto TCP (6), length 57) 217.69.139.74.110 > 172.16.1.3.40904: P, cksum 0x88de (correct), 1:6(5) ack 1 win 5592 <nop,nop,timestamp 891498965 490180>
IP (tos 0x0, ttl 54, id 7311, offset 0, flags [DF], proto TCP (6), length 57) 217.69.139.74.110 > 10.0.3.1.40904: P, cksum 0x28f1 (correct), 1:6(5) ack 1 win 5592 <nop,nop,timestamp 891498965 490180>
IP (tos 0x10, ttl 64, id 35731, offset 0, flags [DF], proto TCP (6), length 52) 10.0.3.1.40904 > 217.69.139.74.110: ., cksum 0xa85d (correct), ack 6 win 5840 <nop,nop,timestamp 490180 891498965>
IP (tos 0x10, ttl 63, id 35731, offset 0, flags [DF], proto TCP (6), length 52) 172.16.1.3.40904 > 217.69.139.74.110: ., cksum 0x084b (correct), ack 6 win 5840 <nop,nop,timestamp 490180 891498965>
IP (tos 0x10, ttl 64, id 35732, offset 0, flags [DF], proto TCP (6), length 52) 10.0.3.1.40904 > 217.69.139.74.110: F, cksum 0xa752 (correct), 1:1(0) ack 6 win 5840 <nop,nop,timestamp 490446 891498965>
IP (tos 0x10, ttl 63, id 35732, offset 0, flags [DF], proto TCP (6), length 52) 172.16.1.3.40904 > 217.69.139.74.110: F, cksum 0x0740 (correct), 1:1(0) ack 6 win 5840 <nop,nop,timestamp 490446 891498965>
IP (tos 0x0, ttl 55, id 7312, offset 0, flags [DF], proto TCP (6), length 52) 217.69.139.74.110 > 172.16.1.3.40904: ., cksum 0xfdd5 (correct), ack 2 win 5592 <nop,nop,timestamp 891501623 490446>
IP (tos 0x0, ttl 54, id 7312, offset 0, flags [DF], proto TCP (6), length 52) 217.69.139.74.110 > 10.0.3.1.40904: ., cksum 0x9de8 (correct), ack 2 win 5592 <nop,nop,timestamp 891501623 490446>
IP (tos 0x0, ttl 55, id 7313, offset 0, flags [DF], proto TCP (6), length 52) 217.69.139.74.110 > 172.16.1.3.40904: F, cksum 0xfdd4 (correct), 6:6(0) ack 2 win 5592 <nop,nop,timestamp 891501623 490446>
IP (tos 0x0, ttl 54, id 7313, offset 0, flags [DF], proto TCP (6), length 52) 217.69.139.74.110 > 10.0.3.1.40904: F, cksum 0x9de7 (correct), 6:6(0) ack 2 win 5592 <nop,nop,timestamp 891501623 490446>
IP (tos 0x0, ttl 64, id 0, offset 0, flags [DF], proto TCP (6), length 52) 10.0.3.1.40904 > 217.69.139.74.110: ., cksum 0x9cef (correct), ack 7 win 5840 <nop,nop,timestamp 490446 891501623>
IP (tos 0x0, ttl 63, id 0, offset 0, flags [DF], proto TCP (6), length 52) 172.16.1.3.40904 > 217.69.139.74.110: ., cksum 0xfcdc (correct), ack 7 win 5840 <nop,nop,timestamp 490446 891501623>
\end{Verbatim}

Выполним с него \textbf{telnet 217.69.139.90 143},  чтобы проверить, что не
происходит подключение к сервисам, например, imap.mail.ru на порту 143:
\begin{Verbatim}
ws13:~# telnet 217.69.139.90 143
Trying 217.69.139.90...
\end{Verbatim}

Вывод \textbf{tcpdump -i any -tnv -p tcp} на \textbf{r1}:
\begin{Verbatim}
IP (tos 0x10, ttl 64, id 19083, offset 0, flags [DF], proto TCP (6), length 60) 10.0.3.1.36591 > 217.69.139.90.143: S, cksum 0xeeab (correct), 4103389992:4103389992(0) win 5840 <mss 1460,sackOK,timestamp 493987 0,nop,wscale 4>
\end{Verbatim}

Выполним с хоста \textbf{ws13} \textbf{traceroute -n 10.102.0.2},  чтобы проверить,
что разрешен vpn:
\begin{Verbatim}
ws13:~# traceroute -n 10.102.0.2
traceroute to 10.102.0.2 (10.102.0.2), 64 hops max, 40 byte packets
 1  10.0.0.1  1 ms  0 ms  0 ms
 2  10.200.1.2  2 ms  1 ms  1 ms
 3  10.102.0.2  1 ms  1 ms  1 ms
\end{Verbatim}

С хоста \textbf{pc21} должны быть доступны любые соединения по vpn и
соединения на порте 80.

Выполним с него \textbf{telnet 213.180.193.3 80},  чтобы проверить, что
происходит подключение к сервисам, например, ya.ru на порту 80:
\begin{Verbatim}
pc21:~# telnet 213.180.193.3 80
Trying 213.180.193.3...
Connected to 213.180.193.3.
Escape character is '^]'.
\end{Verbatim}

Вывод \textbf{tcpdump -i any -tnv -p tcp} на \textbf{r1}:
\begin{Verbatim}
IP (tos 0x10, ttl 64, id 43455, offset 0, flags [DF], proto TCP (6), length 60) 10.102.0.2.48579 > 213.180.193.3.80: S, cksum 0xed9e (correct), 437924172:437924172(0) win 5840 <mss 1460,sackOK,timestamp 497907 0,nop,wscale 4>
IP (tos 0x10, ttl 63, id 43455, offset 0, flags [DF], proto TCP (6), length 60) 172.16.1.4.48579 > 213.180.193.3.80: S, cksum 0x4af2 (correct), 437924172:437924172(0) win 5840 <mss 1460,sackOK,timestamp 497907 0,nop,wscale 4>
IP (tos 0x0, ttl 55, id 0, offset 0, flags [DF], proto TCP (6), length 52) 213.180.193.3.80 > 172.16.1.4.48579: S, cksum 0x5a03 (correct), 1358381019:1358381019(0) ack 437924173 win 14100 <mss 1410,nop,nop,sackOK,nop,wscale 9>
IP (tos 0x0, ttl 54, id 0, offset 0, flags [DF], proto TCP (6), length 52) 213.180.193.3.80 > 10.102.0.2.48579: S, cksum 0xfcaf (correct), 1358381019:1358381019(0) ack 437924173 win 14100 <mss 1410,nop,nop,sackOK,nop,wscale 9>
IP (tos 0x10, ttl 64, id 43456, offset 0, flags [DF], proto TCP (6), length 40) 10.102.0.2.48579 > 213.180.193.3.80: ., cksum 0x72f9 (correct), ack 1 win 365
IP (tos 0x10, ttl 63, id 43456, offset 0, flags [DF], proto TCP (6), length 40) 172.16.1.4.48579 > 213.180.193.3.80: ., cksum 0xd04c (correct), ack 1 win 365
IP (tos 0x10, ttl 64, id 43457, offset 0, flags [DF], proto TCP (6), length 40) 10.102.0.2.48579 > 213.180.193.3.80: F, cksum 0x72f8 (correct), 1:1(0) ack 1 win 365
IP (tos 0x10, ttl 63, id 43457, offset 0, flags [DF], proto TCP (6), length 40) 172.16.1.4.48579 > 213.180.193.3.80: F, cksum 0xd04b (correct), 1:1(0) ack 1 win 365
IP (tos 0x0, ttl 55, id 19526, offset 0, flags [DF], proto TCP (6), length 40) 213.180.193.3.80 > 172.16.1.4.48579: F, cksum 0xd19b (correct), 1:1(0) ack 2 win 28
IP (tos 0x0, ttl 54, id 19526, offset 0, flags [DF], proto TCP (6), length 40) 213.180.193.3.80 > 10.102.0.2.48579: F, cksum 0x7448 (correct), 1:1(0) ack 2 win 28
IP (tos 0x10, ttl 64, id 43458, offset 0, flags [DF], proto TCP (6), length 40) 10.102.0.2.48579 > 213.180.193.3.80: ., cksum 0x72f7 (correct), ack 2 win 365
IP (tos 0x10, ttl 63, id 43458, offset 0, flags [DF], proto TCP (6), length 40) 172.16.1.4.48579 > 213.180.193.3.80: ., cksum 0xd04a (correct), ack 2 win 365
\end{Verbatim}

Выполним с него \textbf{telnet 217.69.139.90 143},  чтобы проверить, что не
происходит подключение к сервисам, например, imap.mail.ru на порту 143:
\begin{Verbatim}
pc21:~# telnet 217.69.139.90 143
Trying 217.69.139.90...
\end{Verbatim}

Вывод \textbf{tcpdump -i any -tnv -p tcp} на \textbf{r1}:
\begin{Verbatim}
IP (tos 0x10, ttl 64, id 12600, offset 0, flags [DF], proto TCP (6), length 60) 10.102.0.2.41859 > 217.69.139.90.143: S, cksum 0x9ecb (correct), 1041078544:1041078544(0) win 5840 <mss 1460,sackOK,timestamp 501800 0,nop,wscale 4>
\end{Verbatim}

Выполним с хоста \textbf{pc21} \textbf{traceroute -n 10.0.1.1},  чтобы проверить,
что разрешен vpn:
\begin{Verbatim}
pc21:~# traceroute -n 10.0.1.1
traceroute to 10.0.1.1 (10.0.1.1), 64 hops max, 40 byte packets
 1  10.102.0.1  0 ms  0 ms  0 ms
 2  10.200.1.1  1 ms  1 ms  1 ms
 3  10.0.1.1  1 ms  1 ms  1 ms
\end{Verbatim}


\section{Проверка доступа к внутреннему серверу}

Проверим правильность найтройки правил фильтрации на маршрутизаторах
с использованием  telnet.

Для проверки доступа к \textbf{s12} выполним со своей машины
\textbf{telnet 172.16.1.3 80}:
\begin{Verbatim}
alex@alex-HP-Pavilion-dv5-Notebook-PC:~/np/lab4$ telnet 172.16.1.3 80
Trying 172.16.1.3...
Connected to 172.16.1.3.
Escape character is '^]'.
\end{Verbatim}

Вывод \textbf{tcpdump -i any -tnv -p tcp} на \textbf{r1}:
\begin{Verbatim}
IP (tos 0x10, ttl 64, id 55991, offset 0, flags [DF], proto TCP (6), length 57) 172.16.1.2.49300 > 172.16.1.3.80: P, cksum 0x88cf (correct), 292697404:292697409(5) ack 2915129968 win 913 <nop,nop,timestamp 1332068 75513>
IP (tos 0x10, ttl 63, id 55991, offset 0, flags [DF], proto TCP (6), length 57) 172.16.1.2.49300 > 10.0.4.20.80: P, cksum 0x27cf (correct), 292697404:292697409(5) ack 2915129968 win 913 <nop,nop,timestamp 1332068 75513>
IP (tos 0x10, ttl 64, id 0, offset 0, flags [DF], proto TCP (6), length 40) 10.0.4.20.80 > 172.16.1.2.49300: R, cksum 0x2ba4 (correct), 2915129968:2915129968(0) win 0
IP (tos 0x10, ttl 63, id 0, offset 0, flags [DF], proto TCP (6), length 40) 172.16.1.3.80 > 172.16.1.2.49300: R, cksum 0x8ca4 (correct), 2915129968:2915129968(0) win 0
IP (tos 0x10, ttl 64, id 21009, offset 0, flags [DF], proto TCP (6), length 60) 172.16.1.2.50934 > 172.16.1.3.80: S, cksum 0x73a5 (correct), 494929698:494929698(0) win 14600 <mss 1460,sackOK,timestamp 1332531 0,nop,wscale 4>
IP (tos 0x10, ttl 63, id 21009, offset 0, flags [DF], proto TCP (6), length 60) 172.16.1.2.50934 > 10.0.4.20.80: S, cksum 0x12a5 (correct), 494929698:494929698(0) win 14600 <mss 1460,sackOK,timestamp 1332531 0,nop,wscale 4>
IP (tos 0x0, ttl 64, id 0, offset 0, flags [DF], proto TCP (6), length 60) 10.0.4.20.80 > 172.16.1.2.50934: S, cksum 0xf679 (correct), 1591093637:1591093637(0) ack 494929699 win 5792 <mss 1460,sackOK,timestamp 505375 1332531,nop,wscale 4>
IP (tos 0x0, ttl 63, id 0, offset 0, flags [DF], proto TCP (6), length 60) 172.16.1.3.80 > 172.16.1.2.50934: S, cksum 0x577a (correct), 1591093637:1591093637(0) ack 494929699 win 5792 <mss 1460,sackOK,timestamp 505375 1332531,nop,wscale 4>
IP (tos 0x10, ttl 64, id 21010, offset 0, flags [DF], proto TCP (6), length 52) 172.16.1.2.50934 > 172.16.1.3.80: ., cksum 0x9952 (correct), ack 1 win 913 <nop,nop,timestamp 1332531 505375>
IP (tos 0x10, ttl 63, id 21010, offset 0, flags [DF], proto TCP (6), length 52) 172.16.1.2.50934 > 10.0.4.20.80: ., cksum 0x3852 (correct), ack 1 win 913 <nop,nop,timestamp 1332531 505375>
IP (tos 0x10, ttl 64, id 21011, offset 0, flags [DF], proto TCP (6), length 52) 172.16.1.2.50934 > 172.16.1.3.80: F, cksum 0x9739 (correct), 1:1(0) ack 1 win 913 <nop,nop,timestamp 1333067 505375>
IP (tos 0x10, ttl 63, id 21011, offset 0, flags [DF], proto TCP (6), length 52) 172.16.1.2.50934 > 10.0.4.20.80: F, cksum 0x3639 (correct), 1:1(0) ack 1 win 913 <nop,nop,timestamp 1333067 505375>
IP (tos 0x0, ttl 64, id 40817, offset 0, flags [DF], proto TCP (6), length 52) 10.0.4.20.80 > 172.16.1.2.50934: F, cksum 0x3787 (correct), 1:1(0) ack 2 win 362 <nop,nop,timestamp 505591 1333067>
IP (tos 0x0, ttl 63, id 40817, offset 0, flags [DF], proto TCP (6), length 52) 172.16.1.3.80 > 172.16.1.2.50934: F, cksum 0x9887 (correct), 1:1(0) ack 2 win 362 <nop,nop,timestamp 505591 1333067>
IP (tos 0x10, ttl 64, id 21012, offset 0, flags [DF], proto TCP (6), length 52) 172.16.1.2.50934 > 172.16.1.3.80: ., cksum 0x965d (correct), ack 2 win 913 <nop,nop,timestamp 1333070 505591>
IP (tos 0x10, ttl 63, id 21012, offset 0, flags [DF], proto TCP (6), length 52) 172.16.1.2.50934 > 10.0.4.20.80: ., cksum 0x355d (correct), ack 2 win 913 <nop,nop,timestamp 1333070 505591>
\end{Verbatim}

Для проверки доступа к \textbf{s11} выполним со своей машины
\textbf{telnet 172.16.1.3 25}:
\begin{Verbatim}
alex@alex-HP-Pavilion-dv5-Notebook-PC:~/np/lab4$ telnet 172.16.1.3 25
Trying 172.16.1.3...
Connected to 172.16.1.3.
Escape character is '^]'.
\end{Verbatim}

Вывод \textbf{tcpdump -i any -tnv -p tcp} на \textbf{r1}:
\begin{Verbatim}
IP (tos 0x10, ttl 64, id 11757, offset 0, flags [DF], proto TCP (6), length 60) 172.16.1.2.48103 > 172.16.1.3.25: S, cksum 0xc0b0 (correct), 169781703:169781703(0) win 14600 <mss 1460,sackOK,timestamp 1344554 0,nop,wscale 4>
IP (tos 0x10, ttl 63, id 11757, offset 0, flags [DF], proto TCP (6), length 60) 172.16.1.2.48103 > 10.0.4.10.25: S, cksum 0x5fba (correct), 169781703:169781703(0) win 14600 <mss 1460,sackOK,timestamp 1344554 0,nop,wscale 4>
IP (tos 0x0, ttl 64, id 0, offset 0, flags [DF], proto TCP (6), length 60) 10.0.4.10.25 > 172.16.1.2.48103: S, cksum 0x1b0a (correct), 2362904855:2362904855(0) ack 169781704 win 5792 <mss 1460,sackOK,timestamp 510225 1344554,nop,wscale 4>
IP (tos 0x0, ttl 63, id 0, offset 0, flags [DF], proto TCP (6), length 60) 172.16.1.3.25 > 172.16.1.2.48103: S, cksum 0x7c00 (correct), 2362904855:2362904855(0) ack 169781704 win 5792 <mss 1460,sackOK,timestamp 510225 1344554,nop,wscale 4>
IP (tos 0x10, ttl 64, id 11758, offset 0, flags [DF], proto TCP (6), length 52) 172.16.1.2.48103 > 172.16.1.3.25: ., cksum 0xbdd7 (correct), ack 1 win 913 <nop,nop,timestamp 1344555 510225>
IP (tos 0x10, ttl 63, id 11758, offset 0, flags [DF], proto TCP (6), length 52) 172.16.1.2.48103 > 10.0.4.10.25: ., cksum 0x5ce1 (correct), ack 1 win 913 <nop,nop,timestamp 1344555 510225>
IP (tos 0x0, ttl 64, id 12177, offset 0, flags [DF], proto TCP (6), length 60) 10.0.4.10.39104 > 172.16.1.2.113: S, cksum 0xa052 (correct), 2365187712:2365187712(0) win 5840 <mss 1460,sackOK,timestamp 510225 0,nop,wscale 4>
IP (tos 0x0, ttl 63, id 12177, offset 0, flags [DF], proto TCP (6), length 60) 172.16.1.3.39104 > 172.16.1.2.113: S, cksum 0x0149 (correct), 2365187712:2365187712(0) win 5840 <mss 1460,sackOK,timestamp 510225 0,nop,wscale 4>
IP (tos 0x0, ttl 64, id 0, offset 0, flags [DF], proto TCP (6), length 40) 172.16.1.2.113 > 172.16.1.3.39104: R, cksum 0x48fe (correct), 0:0(0) ack 2365187713 win 0
IP (tos 0x0, ttl 63, id 0, offset 0, flags [DF], proto TCP (6), length 40) 172.16.1.2.113 > 10.0.4.10.39104: R, cksum 0xe807 (correct), 0:0(0) ack 2365187713 win 0
IP (tos 0x10, ttl 64, id 11759, offset 0, flags [DF], proto TCP (6), length 52) 172.16.1.2.48103 > 172.16.1.3.25: F, cksum 0xbc08 (correct), 1:1(0) ack 1 win 913 <nop,nop,timestamp 1345017 510225>
IP (tos 0x10, ttl 63, id 11759, offset 0, flags [DF], proto TCP (6), length 52) 172.16.1.2.48103 > 10.0.4.10.25: F, cksum 0x5b12 (correct), 1:1(0) ack 1 win 913 <nop,nop,timestamp 1345017 510225>
IP (tos 0x0, ttl 64, id 55305, offset 0, flags [DF], proto TCP (6), length 52) 10.0.4.10.25 > 172.16.1.2.48103: ., cksum 0x5c7f (correct), ack 2 win 362 <nop,nop,timestamp 510411 1345017>
IP (tos 0x0, ttl 63, id 55305, offset 0, flags [DF], proto TCP (6), length 52) 172.16.1.3.25 > 172.16.1.2.48103: ., cksum 0xbd75 (correct), ack 2 win 362 <nop,nop,timestamp 510411 1345017>
\end{Verbatim}

\end{document}
