\documentclass[a4paper,12pt]{article}

\input{header.tex}

\title{Отчёт по лабораторной работе \\ <<Локальные сети>>}
\author{Фроловский Алексей Вадимлвич}

\begin{document}

\maketitle

\tableofcontents

% Текст отчёта должен быть читаемым!!! Написанное здесь является рыбой.

\section{Получение адреса по DHCP}

Получение "случайного" адреса можно продемонстрировать на примере
сети \textbf{lan2} с адресом \textbf{10.102.0.0/16}. Для этого
необходимо настроить службу DHCP на маршрутизаторе \textbf{r2}:
в файле \textbf{/etc/dhcp3/dhcpd.conf} следует указать следующие
настройки:
\begin{Verbatim}
subnet 172.16.0.0 netmask 255.255.0.0 \{\}

subnet 10.102.0.0 netmask 255.255.0.0
\{
  range 10.102.0.2 10.102.0.200;
  option routers 10.102.0.1;
  option domain-name-servers 192.168.0.1;
\}
\end{Verbatim}

Отметим, что в настройке \textbf{range} указывается диапазон,
в котором будут динамически выдаваться ip-адреса, в
\textbf{option routers} - ip-адреса маршрутизаторов сети,
\textbf{option domain-name-servers} - ip-адреса DNS-серверов.

Выполним на маршрутизаторе \textbf{r2}:
\begin{Verbatim}
tcpdump -tnv -i eth0 udp
\end{Verbatim}

Получим следующий вывод:
\begin{Verbatim}
IP (tos 0x10, ttl 128, id 0, offset 0, flags [none], proto UDP (17), length 328) 
	0.0.0.0.68 > 255.255.255.255.67: BOOTP/DHCP, Request from 10:10:10:10:10:ee, length 300, xid 0x82701421, Flags [none]
          Client-Ethernet-Address 10:10:10:10:10:ee [|bootp]
IP (tos 0x10, ttl 128, id 0, offset 0, flags [none], proto UDP (17), length 328) 
	10.102.0.1.67 > 10.102.0.2.68: BOOTP/DHCP, Reply, length 300, xid 0x82701421, Flags [none]
          Your-IP 10.102.0.2
          Client-Ethernet-Address 10:10:10:10:10:ee [|bootp]
IP (tos 0x10, ttl 128, id 0, offset 0, flags [none], proto UDP (17), length 328) 
	0.0.0.0.68 > 255.255.255.255.67: BOOTP/DHCP, Request from 10:10:10:10:10:ee, length 300, xid 0x82701421, Flags [none]
          Client-Ethernet-Address 10:10:10:10:10:ee [|bootp]
IP (tos 0x10, ttl 128, id 0, offset 0, flags [none], proto UDP (17), length 328) 
	10.102.0.1.67 > 10.102.0.2.68: BOOTP/DHCP, Reply, length 300, xid 0x82701421, Flags [none]
          Your-IP 10.102.0.2
          Client-Ethernet-Address 10:10:10:10:10:ee [|bootp]
IP (tos 0x0, ttl 64, id 0, offset 0, flags [DF], proto UDP (17), length 328) 
	10.102.0.2.68 > 10.102.0.1.67: BOOTP/DHCP, Request from 10:10:10:10:10:ee, length 300, xid 0x7185b80d, Flags [none]
          Client-IP 10.102.0.2
          Client-Ethernet-Address 10:10:10:10:10:ee [|bootp]
IP (tos 0x10, ttl 128, id 0, offset 0, flags [none], proto UDP (17), length 328) 
	0.0.0.0.68 > 255.255.255.255.67: BOOTP/DHCP, Request from 10:10:10:10:10:ee, length 300, xid 0xe433c82b, Flags [none]
          Client-Ethernet-Address 10:10:10:10:10:ee [|bootp]
IP (tos 0x10, ttl 128, id 0, offset 0, flags [none], proto UDP (17), length 328) 10.102.0.1.67 > 10.102.0.2.68: BOOTP/DHCP, Reply, length 300, xid 0xe433c82b, Flags [none]
          Your-IP 10.102.0.2
          Client-Ethernet-Address 10:10:10:10:10:ee [|bootp]
IP (tos 0x10, ttl 128, id 0, offset 0, flags [none], proto UDP (17), length 328) 
	0.0.0.0.68 > 255.255.255.255.67: BOOTP/DHCP, Request from 10:10:10:10:10:ee, length 300, xid 0xe433c82b, Flags [none]
          Client-Ethernet-Address 10:10:10:10:10:ee [|bootp]
IP (tos 0x10, ttl 128, id 0, offset 0, flags [none], proto UDP (17), length 328) 
	10.102.0.1.67 > 10.102.0.2.68: BOOTP/DHCP, Reply, length 300, xid 0xe433c82b, Flags [none]
          Your-IP 10.102.0.2
          Client-Ethernet-Address 10:10:10:10:10:ee [|bootp]
\end{Verbatim}

Получение "фиксированных" адресов можно продемонстрировать на
примере сети \textbf{lan1} с адресом \textbf{10.0.0.0/16}. Для
этого необходимо настроить службу DHCP на маршрутизаторе
\textbf{r1}: в файле \textbf{/etc/dhcp3/dhcpd.conf} следует
указать следующие настройки:
\begin{Verbatim}
subnet 172.16.0.0 netmask 255.255.0.0 \{\}

subnet 10.0.0.0 netmask 255.255.0.0
\{
  range 10.0.0.2 10.0.10.200;
  option routers 10.0.0.1;
  option domain-name-servers 192.168.0.1;
\}

host ws11 \{
    hardware ethernet 10:10:10:10:10:BA;
    fixed-address 10.0.1.1;
\}

host ws12 \{
    hardware ethernet 10:10:10:10:10:BB;
    fixed-address 10.0.2.1;
\}

host ws13 \{
    hardware ethernet 10:10:10:10:10:BC;
    fixed-address 10.0.3.1;
\}

host s11 \{
    hardware ethernet 10:10:10:10:20:AA;
    fixed-address 10.0.4.10;
\}

host s12 \{
    hardware ethernet 10:10:10:10:20:BB;
    fixed-address 10.0.4.20;
\}
\end{Verbatim}

Где что дампим.

\begin{Verbatim}
получение "фиксированого" адреса
\end{Verbatim}


\section{Использование VPN}

\begin{Verbatim}
ip r на маршрутизаторе после VPN и работы RIP
\end{Verbatim}

\begin{Verbatim}
ip -4 a  на маршрутизаторе
\end{Verbatim}

\begin{Verbatim}
просшулка сообщений RIP на tun0
\end{Verbatim}


\section{Правила фильтации пакетов и трансляции пдресов}

Где что дампим.

\begin{Verbatim}
сценарий фильтрации
\end{Verbatim}

\begin{Verbatim}
iptables -L -nv
\end{Verbatim}

\begin{Verbatim}
iptables -L -nv -t nat
\end{Verbatim}

\section{Проверка трансляции}

SNAT

\begin{Verbatim}
дамп SNAT в LAN
\end{Verbatim}

\begin{Verbatim}
дамп SNAT (снаружи)
\end{Verbatim}

DNAT

\begin{Verbatim}
дамп DNAT (снаружи)
\end{Verbatim}

\begin{Verbatim}
дамп DNAT в LAN
\end{Verbatim}


\section{Проверка правил фильтрации}

Используем telnet.

\section{Проверка доступа к внутреннему серверу}

Используем telnet.

\end{document}
