\documentclass[a4paper,12pt]{article}

\usepackage[utf8x]{inputenc}
\usepackage[T2A]{fontenc}
\usepackage[english, russian]{babel}

% Опционно, требует  apt-get install scalable-cyrfonts.*
% и удаления одной строчки в cyrtimes.sty
% Сточку не удалять!
% \usepackage{cyrtimes}

% Картнки и tikz
\usepackage{graphicx}
\usepackage{tikz}
\usetikzlibrary{snakes,arrows,shapes}


% Некоторая русификация.
\usepackage{misccorr}
\usepackage{indentfirst}
\renewcommand{\labelitemi}{\normalfont\bfseries{--}}

% Увы, поля придётся уменьшить из-за листингов.
\topmargin -1cm
\oddsidemargin -0.5cm
\evensidemargin -0.5cm
\textwidth 17cm
\textheight 24cm

\sloppy

% Оглавление в PDF
\usepackage[
bookmarks=true,
colorlinks=true, linkcolor=black, anchorcolor=black, citecolor=black, menucolor=black,filecolor=black, urlcolor=black,
unicode=true
]{hyperref}

% Для исходного кода в тексте
\newcommand{\Code}[1]{\texttt{#1}}


\title{Отчёт по лабораторной работе \\ <<Механизмы протокола TCP>>}
\author{Здесь Ф.~И.~О}

\begin{document}

\maketitle

\tableofcontents

% Текст отчёта должен быть читаемым!!! Написанное здесь является рыбой.

\section{Установка и разрыв соединения}

Где что дампим.

\begin{Verbatim}
дамп с syn + mss clamp + F
\end{Verbatim}

\section{Окно получателя}

Где что дампим.  Дампить без -t обязательно!

\begin{Verbatim}
окно получателя до нуля
\end{Verbatim}

\section{Окно отправителя}

Где что дампим.

\begin{Verbatim}
окно отправителя растёт и растёт
\end{Verbatim}

\section{Нейгл и Мишналь}

Где что дампим.  Дампить без -t обязательно!

\begin{Verbatim}
увидеть хотя бы нейгла
\end{Verbatim}

\section{Аггрессивная буферизация}

Где что дампим.  Дампить без -t обязательно!

\begin{Verbatim}
cork
\end{Verbatim}

\section{Отправка без задержки}

Где что дампим.  Дампить без -t обязательно!

\begin{Verbatim}
увидеть, что nodelay не помогает, когда окно отправителя полное
\end{Verbatim}

\section{Быстрый повтор}

Где что дампим.  Дампить без -t обязательно!

\begin{Verbatim}
увидеть быстрый повтор
\end{Verbatim}

\section{Обычный повтор}

Где что дампим. Дампить без -t обязательно!

\begin{Verbatim}
увидеть не-быстрый повтор
\end{Verbatim}

\section{Неудачная попытка соединени с портом}

Тут опт с попыткой соединения с портом, который никто не слушает.

\section{Опыт с PMTU}

Для опыта нужно на c3 и c4 отключить tcpclump.
Надеюсь тут поможет команда iptables -F.
Для сброса кеша MSS на с1 его можно тупо перегрузить.

После этих подготовительных действий мы наверное увидим pmtu в действии при попытке соединится с c1 на c6.

\begin{Verbatim}
Дампить (tcp or icmp) на c2 eth0.
\end{Verbatim}

\section{Соединение с неверным портом}

Что будет, если клиент пытается соединиться с портом, который не слушвет сервер.

\begin{Verbatim}
Дампить
\end{Verbatim}

\end{document}
